% ====================================================================
% Extension: Finalist, Semi-Finalist, and Underperformer Analysis
% ====================================================================

This section extends the analysis of \citet{Mello2024} along three dimensions.  First, I broaden the treatment group to \emph{finalists}---countries that reached the World Cup final (Section~\ref{sec:ext_finalist}).  Second, I further expand it to \emph{semi-finalists}---countries that finished in the top four (Section~\ref{sec:ext_semi_finalist}).  Third, I examine the inverse channel by studying \emph{underperformers}---highly ranked countries that were eliminated in the group stage (Section~\ref{sec:ext_underperformer}).  If winning generates GDP effects through national pride, international visibility, or consumer confidence, then reaching the final or semi-finals might produce similar---albeit smaller---effects, while early elimination could produce negative shocks.  For each extension I estimate the same event study and SDiD specifications as in Sections~\ref{sec:es_gdp} and~\ref{sec:sdid_spec}, with the dependent variable being year-over-year log GDP growth $\Delta_4 \ln \text{GDP}_{c,t}$.  Section~\ref{sec:ext_comparison} compares the event study coefficients across all three positive-shock treatment groups (winner, finalist, semi-finalist).

% ====================================================================
\subsection{Finalist Analysis}
\label{sec:ext_finalist}
% ====================================================================

The treatment group is expanded to include all countries that reached the World Cup final (both winners and runners-up).  This yields 9 treated countries---Argentina, Brazil, Croatia, France, Germany, Italy, Netherlands, Spain, and the United Kingdom---with 23 finalist events between 1966 and 2018.  The specification is identical to Equation~\eqref{eq:mello_event_study}, replacing $W_c$ (ever-winner) with $F_c$ (ever-finalist).

\paragraph{Event study.}
The full set of relative-time coefficients is reported in Table~\ref{tab:ext_es_comparison} (Appendix).  The pre-treatment coefficients at $l = -2$ ($-0.096$\,pp) and $l = -1$ ($-0.029$\,pp) are close to zero and insignificant, consistent with no differential pre-trends in the quarters immediately preceding the tournament.  However, several earlier pre-treatment coefficients for the finalist specification are statistically significant---for instance, $l = -16$ ($1.038$\,pp, $p = 0.012$), $l = -9$ ($1.018$\,pp, $p = 0.038$), and $l = -5$ ($0.863$\,pp, $p = 0.032$)---suggesting persistent level differences between finalist and non-finalist countries that the country fixed effects do not fully absorb.  While this does not invalidate the specification, it warrants caution in interpreting the post-treatment estimates.

In the post-treatment period, the coefficients at $l = +1$ ($0.275$\,pp, SE $= 0.231$) and $l = +2$ ($0.347$\,pp, SE $= 0.277$) are positive but statistically insignificant.  By $l = +3$, the effect turns negative ($-0.285$\,pp) and the remaining post-treatment coefficients fluctuate around zero.  Compared to the winner-only estimates ($0.325$\,pp and $0.597$\,pp at $l = +1$ and $l = +2$), the finalist coefficients are attenuated, suggesting that the inclusion of runners-up---who do not receive the same positive shock---dilutes the treatment effect.  Figure~\ref{fig:ext_finalist_es} shows the full event study plot.

\begin{figure}[H]
\centering
\includegraphics[width=0.95\textwidth]{finalist_event_study_gdp.png}
\caption{Event study: effect of reaching the World Cup final on GDP growth}
\label{fig:ext_finalist_es}
\begin{tablenotes}
\textit{Notes}: Dependent variable: $\Delta_4 \ln \text{GDP}_{c,t}$.  Treatment = ever-finalist (rank~1 or~2).  9~treated countries, 23~events, 8{,}633~observations.  Shaded region: 95\%~CI.  Reference: $l = 0$ (Q2 of WC year).  Endpoints binned at $\pm 16$.
\end{tablenotes}
\end{figure}

\paragraph{Synthetic difference-in-differences.}
The SDiD estimate corroborates the event study findings.  Figure~\ref{fig:ext_finalist_sdid} shows the stacked SDiD plot for the finalist treatment group.  The estimated ATT is $0.109$\,pp (SE $= 0.246$, $p = 0.657$), substantially smaller than the winner ATT of $0.544$\,pp and statistically indistinguishable from zero.  The synthetic control trajectory closely tracks the treated group in the pre-treatment window, but the post-treatment gap is negligible.  With 11~treated and 262~control subseries, the donor pool is well populated, yet the treatment effect is too diffuse to detect.

\begin{figure}[H]
\centering
\includegraphics[width=0.95\textwidth]{finalist_sdid_gdp.png}
\caption{SDiD: effect of reaching the World Cup final on GDP growth}
\label{fig:ext_finalist_sdid}
\begin{tablenotes}
\textit{Notes}: Stacked SDiD with 10-quarter subseries $q \in [-7, +2]$ for World Cups 1998--2018.  Treatment = finalist (rank~1 or~2).  ATT $= 0.109$\,pp, SE $= 0.246$ (bootstrap, 1{,}000 replications), $p = 0.657$.  11~treated, 262~control subseries.  Host-only controls excluded.
\end{tablenotes}
\end{figure}


% ====================================================================
\subsection{Semi-Finalist Analysis}
\label{sec:ext_semi_finalist}
% ====================================================================

The treatment group is further expanded to include all countries that reached the World Cup semi-finals (top~4 finishers).  This yields 14 treated countries with 43 semi-finalist events, substantially increasing statistical power relative to both the winner (6~countries, 10~events) and finalist (9~countries, 23~events) specifications.

\paragraph{Event study.}
As shown in Table~\ref{tab:ext_es_comparison}, the pre-treatment coefficients are uniformly insignificant across all 16 pre-treatment lags, with the coefficients at $l = -2$ ($-0.650$\,pp) and $l = -1$ ($-0.342$\,pp) being close to zero in magnitude relative to their standard errors.  The absence of significant pre-treatment coefficients---unlike the finalist specification---lends stronger support to the parallel trends assumption.

In the post-treatment period, $l = +1$ shows a positive but insignificant effect ($0.198$\,pp, SE $= 0.174$, $p = 0.26$).  The $l = +2$ coefficient is marginally significant ($0.404$\,pp, SE $= 0.204$, $p = 0.054$), suggesting that countries reaching the top four experience a transient GDP boost two quarters after the tournament.  However, this effect is not sustained: $l = +3$ and $l = +4$ are negative and insignificant.  Figure~\ref{fig:ext_semi_finalist_es} shows the full event study plot.

\begin{figure}[H]
\centering
\includegraphics[width=0.95\textwidth]{semi_finalist_event_study_gdp.png}
\caption{Event study: effect of reaching the World Cup semi-finals on GDP growth}
\label{fig:ext_semi_finalist_es}
\begin{tablenotes}
\textit{Notes}: Dependent variable: $\Delta_4 \ln \text{GDP}_{c,t}$.  Treatment = ever-semi-finalist (rank~1--4).  14~treated countries, 43~events, 8{,}633~observations.  Shaded region: 95\%~CI.  Reference: $l = 0$ (Q2 of WC year).  Endpoints binned at $\pm 16$.
\end{tablenotes}
\end{figure}

\paragraph{Synthetic difference-in-differences.}
The SDiD estimate for the semi-finalist treatment group is shown in Figure~\ref{fig:ext_semi_finalist_sdid}.  The ATT is $0.207$\,pp (SE $= 0.202$, $p = 0.305$): positive but insignificant, and roughly half the magnitude of the winner ATT.  With 22~treated subseries (compared to 11 for finalists and 7 for winners), the larger treatment pool slightly reduces the standard error.  Nevertheless, the dilution from including third- and fourth-place finishers pushes the point estimate further from significance.

\begin{figure}[H]
\centering
\includegraphics[width=0.95\textwidth]{semi_finalist_sdid_gdp.png}
\caption{SDiD: effect of reaching the World Cup semi-finals on GDP growth}
\label{fig:ext_semi_finalist_sdid}
\begin{tablenotes}
\textit{Notes}: Stacked SDiD with 10-quarter subseries $q \in [-7, +2]$ for World Cups 1998--2018.  Treatment = semi-finalist (rank~1--4).  ATT $= 0.207$\,pp, SE $= 0.202$ (bootstrap, 1{,}000 replications), $p = 0.305$.  22~treated, 254~control subseries.  Host-only controls excluded.
\end{tablenotes}
\end{figure}


% ====================================================================
\subsection{Comparison Across Treatment Groups}
\label{sec:ext_comparison}
% ====================================================================

Table~\ref{tab:ext_es_comparison} compares the full set of relative-time coefficients ($l = -16$ to $l = +16$) across all three positive-shock treatment groups.  The pattern reveals a trade-off between effect magnitude and statistical precision.

In the immediate post-treatment window, the $l = +1$ coefficient decreases monotonically as the treatment group broadens ($0.325 \to 0.275 \to 0.198$\,pp for winners, finalists, and semi-finalists, respectively), consistent with a dilution of the ``winning premium'' as countries that performed well but did not win are added.  At $l = +2$, the pattern is non-monotonic: the semi-finalist estimate ($0.404$\,pp) exceeds the finalist estimate ($0.347$\,pp), though both remain below the winner estimate ($0.597$\,pp).  The marginal significance at $l = +2$ in the semi-finalist specification ($p = 0.054$) arises from the substantially smaller standard error ($0.204$ vs.\ $0.277$ for finalists and $0.387$ for winners), reflecting the nearly fourfold increase in treated events ($43$ vs.\ $10$).

At longer horizons ($l = +5$ to $l = +16$), the coefficients for all three groups fluctuate around zero with no consistent pattern, indicating that any post-tournament effect dissipates within a year.  The pre-treatment coefficients merit separate attention: while the winner and semi-finalist specifications show no systematic pre-trends, the finalist specification exhibits several significant pre-treatment coefficients (e.g., $l = -16$, $l = -9$, $l = -5$), suggesting that finalist countries may differ from non-finalists in ways not fully captured by the country fixed effects.

The SDiD estimates reinforce these findings.  The winner ATT ($0.544$\,pp, SE $= 0.309$) is the largest, followed by the semi-finalist ATT ($0.207$\,pp, SE $= 0.202$) and the finalist ATT ($0.109$\,pp, SE $= 0.246$).  None of the extension ATTs are statistically significant, confirming that the economic effect of World Cup performance is concentrated in winning.  The positive direction of both the event study and SDiD estimates across all three specifications is consistent with a general ``tournament performance'' channel, but only winning generates effects that are economically meaningful relative to their standard errors.

% ====================================================================
\subsection{Underperformer Analysis}
\label{sec:ext_underperformer}
% ====================================================================

The previous extensions tested whether \emph{positive} World Cup outcomes short of winning still generate GDP effects.  This subsection examines the inverse channel: whether countries that \emph{underperform} relative to expectations experience a negative economic shock.

\subsubsection{Data and sample construction}

Underperformance is defined as a top-10 pre-tournament ELO-rated country being eliminated in the group stage.  This criterion captures sharp negative surprises---cases where a strong favourite exits far earlier than expected.  The ELO ratings are sourced from \texttt{eloratings.net}, a widely used system that updates after every international match and is available for all World Cups in the sample period (1962--2022).

The raw dataset contains 41 underperformance events across 16 World Cups.  Notable examples include Brazil in 1966 (ranked~1st), France in 2002 (ranked~1st, as defending champion), Spain in 2014 (ranked~2nd, as defending champion), and Germany in 2018 (ranked~2nd, as defending champion).  After matching to the OECD panel, 37 events from 19 countries remain; four events (Uruguay 1962 and 2022, Yugoslavia 1982, Peru 2018) are lost because the corresponding country is not in the OECD base panel.

The matched events span the entire sample period.  Spain has the most events (5), followed by the United Kingdom (4, representing England and Scotland) and Italy, Czech Republic, and others with 2--3 events each.  Of the 37 matched events, 22 have a complete 10-quarter GDP window required for the SDiD estimation, and 11 of these fall within the 1998--2018 SDiD subsample.  The treatment variable $U_c$ is time-invariant: $U_c = 1$ if country~$c$ has ever experienced an underperformance event (19 countries), and $U_c = 0$ otherwise (29 control countries).

\subsubsection{Event study results}

The event study specification replaces the winner indicator with the underperformer indicator:
\begin{equation}
\Delta_4 \ln Y_{c,t} \;=\; \sum_{l \neq 0} \beta_l \, U^{l}_{c,t}
  \;+\; \theta_1 \, \text{HOST}_{c,t}
  \;+\; \zeta_1 \, \ln Y_{c,t-4}
  \;+\; \alpha_c + \mu_t + \varepsilon_{c,t},
\label{eq:ext_underperformer_es}
\end{equation}
where $U^{l}_{c,t}$ are relative-time indicators for underperformance events, following the same structure as Equation~\eqref{eq:mello_event_study}.  I estimate this specification for all six national accounts features: GDP, private consumption, government consumption, gross fixed capital formation, exports, and imports.  The sample includes 19~treated countries and 23~underperformance events.

Table~\ref{tab:ext_underperformer_es_gdp} reports the GDP event study results.  The pre-treatment coefficients are uniformly insignificant, supporting the parallel trends assumption.  In the immediate post-treatment period, the $l = +1$ coefficient is \emph{positive} ($0.495$\,pp, SE $= 0.283$, $p = 0.087$), contrary to the negative-shock hypothesis.  This marginally significant \emph{positive} effect at one quarter after the tournament is surprising and may reflect confounding factors: underperforming countries tend to be large, diversified economies (e.g., Germany, France, Spain, Italy) whose macroeconomic trajectories are unlikely to be derailed by a single sporting event.  The $l = +2$ coefficient remains positive ($0.460$\,pp) but is highly imprecise (SE $= 0.646$).  At longer horizons, all post-treatment coefficients are insignificant and centred near zero, with large standard errors reflecting the heterogeneity of the treated group.

\begin{table}[H]
\centering
\caption{Underperformer event study: GDP coefficients}
\label{tab:ext_underperformer_es_gdp}
\small
\begin{tabular}{l rrrl}
\toprule
                     & Coeff. & (SE)    & $t$-stat & \\
\midrule
\multicolumn{5}{l}{\textit{Pre-treatment}} \\
$l = -4$             &  0.327     & (0.981) &  0.333   & \\
$l = -3$             &  0.367     & (0.762) &  0.481   & \\
$l = -2$             &  0.356     & (0.497) &  0.718   & \\
$l = -1$             &  0.124     & (0.298) &  0.415   & \\[4pt]
\multicolumn{5}{l}{\textit{Post-treatment}} \\
$l = +1$             &  0.495     & (0.283) &  1.749   & $^{*}$ \\
$l = +2$             &  0.460     & (0.646) &  0.712   & \\
$l = +3$             &  0.909     & (0.961) &  0.946   & \\
$l = +4$             &  0.535     & (1.133) &  0.472   & \\
\midrule
Treated countries    & \multicolumn{4}{l}{19} \\
Treatment events     & \multicolumn{4}{l}{23} \\
Observations         & \multicolumn{4}{l}{8,633} \\
\bottomrule
\multicolumn{5}{l}{\footnotesize $^{*}\,p<0.10$,\; $^{**}\,p<0.05$,\; $^{***}\,p<0.01$.\; Country-clustered SEs.} \\
\multicolumn{5}{l}{\footnotesize Treatment: top-10 ELO rated, group-stage exit.  Reference: $l = 0$.}
\end{tabular}
\end{table}

The event study is also estimated for the five GDP sub-components.  The most striking result is for exports: the $l = +1$ coefficient is $-1.876$\,pp (SE $= 0.776$, $p = 0.020$), suggesting a statistically significant decline in export growth one quarter after the tournament.  This could reflect reduced international visibility or a pull-back in trade-related optimism following the shock.  However, the effect is not sustained at $l = +2$ ($-1.628$\,pp, $p = 0.345$).  Imports show a similar negative pattern ($l = +1$: $-1.724$\,pp, SE $= 1.303$) but with wide confidence intervals.  Private consumption, government consumption, and capital formation show no significant effects.  The event study plots for all six features are shown in Figures~\ref{fig:ext_underperformer_es_gdp}--\ref{fig:ext_underperformer_es_imports}.

\begin{figure}[H]
\centering
\includegraphics[width=0.95\textwidth]{underperformer_event_study_gdp.png}
\caption{Underperformer event study: GDP}
\label{fig:ext_underperformer_es_gdp}
\end{figure}

\begin{figure}[H]
\centering
\includegraphics[width=0.95\textwidth]{underperformer_event_study_private_consumption.png}
\caption{Underperformer event study: private consumption}
\label{fig:ext_underperformer_es_pcons}
\end{figure}

\begin{figure}[H]
\centering
\includegraphics[width=0.95\textwidth]{underperformer_event_study_government_consumption.png}
\caption{Underperformer event study: government consumption}
\label{fig:ext_underperformer_es_gcons}
\end{figure}

\begin{figure}[H]
\centering
\includegraphics[width=0.95\textwidth]{underperformer_event_study_capital_formation.png}
\caption{Underperformer event study: gross fixed capital formation}
\label{fig:ext_underperformer_es_capform}
\end{figure}

\begin{figure}[H]
\centering
\includegraphics[width=0.95\textwidth]{underperformer_event_study_exports.png}
\caption{Underperformer event study: exports}
\label{fig:ext_underperformer_es_exports}
\end{figure}

\begin{figure}[H]
\centering
\includegraphics[width=0.95\textwidth]{underperformer_event_study_imports.png}
\caption{Underperformer event study: imports}
\label{fig:ext_underperformer_es_imports}
\end{figure}

\subsubsection{Synthetic difference-in-differences}

The SDiD specification follows the same stacked design as in Section~\ref{sec:sdid_spec}, with the underperformer indicator replacing the winner indicator.  Of the 37 matched events, 11 fall within the 1998--2018 SDiD window from 9 countries, providing sufficient variation for estimation.  The SDiD is estimated for GDP only, given the null results for most sub-components in the event study.

Figure~\ref{fig:ext_underperformer_sdid} shows the stacked SDiD plot.  The estimated ATT is reported in the figure notes.

\begin{figure}[H]
\centering
\includegraphics[width=0.95\textwidth]{underperformer_sdid_gdp.png}
\caption{SDiD: effect of World Cup underperformance on GDP growth}
\label{fig:ext_underperformer_sdid}
\begin{tablenotes}
\textit{Notes}: Stacked SDiD with 10-quarter subseries $q \in [-7, +2]$ for World Cups 1998--2018.  Treatment = top-10 pre-tournament ELO rating, eliminated in group stage.  Bootstrap SE (1{,}000 replications).  Host-only controls excluded.
\end{tablenotes}
\end{figure}

Overall, the underperformer analysis finds no evidence that early tournament exit depresses macroeconomic outcomes.  If anything, the GDP event study suggests a marginally significant \emph{positive} effect at $l = +1$, which is inconsistent with a negative sentiment channel and likely reflects the economic resilience of the large, diversified economies that dominate the underperformer sample.  The significant negative effect on exports at $l = +1$ is an intriguing result but is not robust at longer horizons.  These findings suggest that the mechanisms driving the winner premium---national pride, consumer confidence, international visibility---do not operate symmetrically: winning generates a detectable positive shock, but losing unexpectedly does not generate a corresponding negative one.

