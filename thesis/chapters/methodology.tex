This section details the empirical implementation of the methods introduced in Section~\ref{theory}. I first describe the replication of \citet{Mello2024}, then present the extensions that constitute the novel contributions of my thesis.

\subsection{Replication of Mello (2024)}
\label{sec:replication_mello}

The primary objective of this thesis is to replicate and compare the analysis in \citet{Mello2024}. As a first step, this involves reconstructing the dataset from OECD sources as the author did not make his code publicly available\footnote{The \citet{Mello2024} paper included a replication package link: \url{https://doi.org/10.3886/E188961} but it leads to a completely different research paper of droughts in Brazil (last visited 02/26).}, reproducing the summary statistics, and estimating both the event study and synthetic difference-in-differences models. The full replication should serve two purposes: validate the original findings using independently constructed data, and additionally establish the foundation for the extensions that follow.
\subsubsection{Data Construction}

The replication uses quarterly national accounts data from the OECD, covering the period 1961--2021. The sample includes 48 countries: 6 World Cup winners during the sample period, 12 host countries, and 36 countries that neither won nor hosted. Following \citet{Mello2024}, Argentina is classified as a control country because its GDP series begins only in 1993---ten years after its last victory in 1986. Brazil enters the treated group starting in 1998-Q2, four years after its 1994 victory, contributing only its 2002 win to the analysis.

The outcome variable is year-over-year (YoY) GDP growth, computed as:
\begin{equation}
\Delta_4 \ln GDP_{c,t} = \ln GDP_{c,t} - \ln GDP_{c,t-4}
\label{eq:yoy_gdp}
\end{equation}
which compares GDP in quarter $t$ to the same quarter one year earlier. This transformation removes seasonal patterns and provides a directly interpretable measure of annual growth at quarterly frequency.

For descriptive statistics and sample composition, see Section~\ref{data}.

\subsubsection{Event Study Specification}

The event study follows the specification in \citet{Mello2024}, which estimates dynamic treatment effects around World Cup victories:
\begin{equation}
\Delta_4 \ln GDP_{c,t} = \alpha_c + \lambda_t + \sum_{k \neq 0} \beta_k \cdot \mathbf{1}[K_{ct} = k] \cdot W_c + \theta \cdot HOST_{c,t} + \zeta \cdot \ln GDP_{c,t-4} + \varepsilon_{c,t}
\label{eq:mello_event_study}
\end{equation}
where $\alpha_c$ are country fixed effects, $\lambda_t$ are quarter fixed effects, $K_{ct} = t - E_c$ is the relative time to country $c$'s World Cup victory at time $E_c$, and $W_c$ is an indicator for ever-winning countries. The reference period is $k = 0$, corresponding to Q2 of the World Cup year. $HOST_{c,t}$ controls for hosting effects in the calendar year of the tournament, and $\ln GDP_{c,t-4}$ controls for the level of economic development four quarters prior.

The relative time indicators span $k \in \{-16, \ldots, -1, +1, \ldots, +16\}$, with quarters beyond this window binned into aggregate endpoints. For countries winning multiple World Cups (Germany: 1974, 1990, 2014; Italy: 1982, 2006; France: 1998, 2018), the relative time counter restarts at the midpoint between consecutive victories.

Standard errors are clustered at the country level to account for serial correlation within countries.

\subsubsection{Synthetic Difference-in-Differences Specification}

The SDID analysis follows \citet{Mello2024} in constructing 10-quarter subseries around each post-1994 World Cup. For each tournament in $\{1998, 2002, 2006, 2010, 2014, 2018\}$, I create subseries spanning $q \in [-7, +2]$ relative to Q2 of the World Cup year. This yields 6 treated subseries (one per winner: France 1998 and 2018, Brazil 2002, Italy 2006, Spain 2010, Germany 2014) and approximately 260 control subseries from the donor pool.

Host country subseries are excluded from the donor pool to avoid contamination, except for France 1998 which both hosted and won. The SDID estimator then solves:
\begin{equation}
\hat{\tau}^{SDID} = \operatorname*{arg\,min}_{\tau, \mu, \alpha, \beta} \sum_{n=1}^{N} \sum_{q=-7}^{2} \left( \Delta_4 \ln GDP_{n,q} - \mu - \alpha_n - \beta_q - W_{n,q} \tau \right)^2 \hat{\omega}_n \hat{\lambda}_q
\label{eq:mello_sdid}
\end{equation}
where $W_{n,q} = 1$ for treated subseries in post-treatment quarters ($q \geq 1$), and the unit weights $\hat{\omega}_n$ and time weights $\hat{\lambda}_q$ are constructed as described in Section~\ref{theory}.

Inference uses 1,000 bootstrap replications with clustering at the country-subseries level, following \citet{Mello2024}.

\subsubsection{GDP Components}

Both the event study and SDID are also estimated separately for GDP components: private consumption, government consumption, gross fixed capital formation, exports, and imports. These decompositions help identify the channels through which World Cup victories affect aggregate output. \citet{Mello2024} finds that the effect is primarily export-driven, which I seek to replicate.

%%%%%%%%%%%%%%%%%%%%%%%%%%%%%%%%%%%%%%%%%%%%%%%%%%%%%%%%%%%%%%%%%%%%%%%%%%%%%%%
\subsection{Extensions: Tournament Performance Effects}
%%%%%%%%%%%%%%%%%%%%%%%%%%%%%%%%%%%%%%%%%%%%%%%%%%%%%%%%%%%%%%%%%%%%%%%%%%%%%%%

Beyond replicating Mello's analysis of World Cup winners, this thesis extends the investigation to other tournament outcomes. If winning generates economic effects through national pride, international visibility, or consumer confidence, then reaching the final or semifinals might produce similar---albeit smaller---effects. Conversely, unexpectedly poor performance might generate negative sentiment effects.

\subsubsection{Finalist Effect}

Countries that reach the World Cup final but lose receive extensive media coverage and experience heightened national attention, similar to winners. To test whether this ``runner-up effect'' generates measurable economic consequences, I estimate:
\begin{equation}
\Delta_4 \ln GDP_{c,t} = \alpha_c + \lambda_t + \sum_{k \neq 0} \beta_k^{F} \cdot \mathbf{1}[K_{ct} = k] \cdot F_c + \theta \cdot HOST_{c,t} + \zeta \cdot \ln GDP_{c,t-4} + \varepsilon_{c,t}
\label{eq:finalist_event_study}
\end{equation}
where $F_c$ indicates countries that reached the final but did not win (e.g., Netherlands 1974, 1978, 2010; Germany 1982, 1986, 2002; Argentina 2014; Croatia 2018). This specification mirrors Mello's approach but with finalists as the treated group.

\subsubsection{Semifinalist Effect}

Reaching the semifinals represents exceptional tournament performance that may generate national pride effects. The specification follows the same structure:
\begin{equation}
\Delta_4 \ln GDP_{c,t} = \alpha_c + \lambda_t + \sum_{k \neq 0} \beta_k^{S} \cdot \mathbf{1}[K_{ct} = k] \cdot S_c + \theta \cdot HOST_{c,t} + \zeta \cdot \ln GDP_{c,t-4} + \varepsilon_{c,t}
\label{eq:semifinalist_event_study}
\end{equation}
where $S_c$ indicates countries reaching the semifinals (finishing in the top 4). This extends the treatment group considerably, providing more statistical power but potentially diluting the effect if only winning matters.

\subsubsection{Early Elimination Effect}

The psychological literature suggests that unexpected negative outcomes can affect sentiment and behavior. A highly-ranked team being eliminated in the group stage or early knockout rounds represents a negative shock to national expectations. To investigate whether such disappointments have measurable economic consequences, I identify cases where pre-tournament favorites (based on FIFA rankings or betting odds) exited earlier than expected.

The specification is:
\begin{equation}
\Delta_4 \ln GDP_{c,t} = \alpha_c + \lambda_t + \sum_{k \neq 0} \beta_k^{E} \cdot \mathbf{1}[K_{ct} = k] \cdot E_c + \theta \cdot HOST_{c,t} + \zeta \cdot \ln GDP_{c,t-4} + \varepsilon_{c,t}
\label{eq:early_exit_event_study}
\end{equation}
where $E_c$ indicates unexpected early elimination. Candidate events include Italy's group stage exit in 2010 (defending champion), Spain's group stage exit in 2014 (defending champion), and Germany's group stage exit in 2018 (defending champion).

This extension tests whether the psychological mechanisms operate symmetrically for positive and negative tournament outcomes.

%%%%%%%%%%%%%%%%%%%%%%%%%%%%%%%%%%%%%%%%%%%%%%%%%%%%%%%%%%%%%%%%%%%%%%%%%%%%%%%
\subsection{Implementation}
%%%%%%%%%%%%%%%%%%%%%%%%%%%%%%%%%%%%%%%%%%%%%%%%%%%%%%%%%%%%%%%%%%%%%%%%%%%%%%%

All analyses are implemented in R. The event study regressions use the \texttt{fixest} package \citep{fixest}, which provides efficient estimation of high-dimensional fixed effects models with clustered standard errors. The SDID estimation uses the \texttt{synthdid} package \citep{synthdid_package}, which implements the \citet{Arkhangelsky2021} estimator with bootstrap inference.

Code and data for full replication are available in the accompanying repository.
