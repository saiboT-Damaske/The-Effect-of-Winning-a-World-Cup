This section describes the empirical methodology used to estimate the causal effect of winning the FIFA World Cup on economic outcomes. The analysis thoroughly replicates the approach of \citet{Mello2024} and then extends it to additional outcome variables, robustness checks, and new cases.

\subsection{Replication Strategy}

The primary objective is to replicate Mello's main results using independently written code and verify that the findings are reproducible. I implement both the event study and SDID methodologies from first principles using R, following the specifications described in the original paper. The replication covers:

\begin{itemize}
\item The main event study regression (Table 2 in Mello)
\item The SDID estimation of the average treatment effect (Figure 1 in Mello)
\item Extension to all GDP components: consumption, investment, exports, and imports
\end{itemize}

Data for this analysis comes from OECD quarterly national accounts, covering 48 countries from 1961-Q1 to 2021-Q4. The sample includes both World Cup winners and a large pool of control countries that never won during the sample period. For detailed information on data sources, variable construction, and sample restrictions, see Section~\ref{data}.

\subsection{Event Study Specification}

The event study regression follows the standard panel data framework with two-way fixed effects. The dependent variable is the year-over-year (YoY) GDP growth rate, defined as:
\begin{equation}
\Delta_4 \ln Y_{it} = \ln Y_{it} - \ln Y_{i,t-4}
\label{eq:yoy_growth}
\end{equation}
where $Y_{it}$ is real GDP (chain-linked volume, PPP-converted to USD) for country $i$ in quarter $t$. This transformation removes seasonal patterns while capturing quarterly dynamics.

The estimated specification is:
\begin{equation}
\Delta_4 \ln GDP_{it} = \alpha_i + \lambda_t + \sum_{k=-16}^{-2} \beta_k \cdot W_{it}^k + \sum_{k=0}^{16} \beta_k \cdot W_{it}^k + \gamma \cdot Host_{it} + \delta \cdot \ln GDP_{i,t-4} + \varepsilon_{it}
\label{eq:event_study_spec}
\end{equation}
where:
\begin{itemize}
\item $W_{it}^k = \mathbf{1}[\text{country } i \text{ is } k \text{ quarters from winning}]$ indicates relative time to World Cup victory
\item $\alpha_i$ are country fixed effects
\item $\lambda_t$ are quarter-year fixed effects
\item $Host_{it}$ is an indicator for whether country $i$ hosted the World Cup in quarter $t$
\item $\ln GDP_{i,t-4}$ is the lagged log GDP level, controlling for convergence dynamics
\item The period $k = -1$ is omitted as the reference category
\end{itemize}

The coefficients $\{\beta_k : k < -1\}$ allow testing for pre-trends---under the parallel trends assumption, these should be statistically indistinguishable from zero. The coefficients $\{\beta_k : k \geq 0\}$ capture the post-victory effects. Standard errors are clustered at the country level to account for serial correlation within countries.

World Cup victories are coded at $k = 0$ for the second quarter (Q2) of each World Cup year, as finals are typically held in late June or early July. The sample includes ten World Cup winners: England (GBR) 1966, Germany (DEU) 1974, Italy (ITA) 1982, Germany (DEU) 1990, France (FRA) 1998, Brazil (BRA) 2002, Italy (ITA) 2006, Spain (ESP) 2010, Germany (DEU) 2014, and France (FRA) 2018.

\subsection{SDID Specification}

For the synthetic difference-in-differences estimation, I follow the stacked panel approach. For each World Cup year $w \in \{1998, 2002, 2006, 2010, 2014, 2018\}$, I create a subseries of observations spanning 8 pre-treatment quarters ($k \in \{-7, \ldots, 0\}$) and 2 post-treatment quarters ($k \in \{1, 2\}$). Each country-World Cup combination forms a unit in the stacked panel.

Treatment is defined as:
\begin{equation}
D_{it} = \mathbf{1}[\text{country } i \text{ won World Cup } w] \times \mathbf{1}[k \geq 1]
\label{eq:sdid_treatment}
\end{equation}

The SDID estimator is then:
\begin{equation}
\hat{\tau}^{SDID} = \left( \frac{1}{N_{tr} \cdot T_{post}} \sum_{i \in \mathcal{T}} \sum_{t > T_0} Y_{it} - \sum_{t \leq T_0} \hat{\lambda}_t Y_{it} \right) - \sum_{j \in \mathcal{C}} \hat{\omega}_j \left( \frac{1}{T_{post}} \sum_{t > T_0} Y_{jt} - \sum_{t \leq T_0} \hat{\lambda}_t Y_{jt} \right)
\label{eq:sdid_implemented}
\end{equation}
where $\mathcal{T}$ is the set of treated units (winners: FRA-1998, BRA-2002, ITA-2006, ESP-2010, DEU-2014, FRA-2018), $\mathcal{C}$ is the set of control units, and $\hat{\omega}_j$ and $\hat{\lambda}_t$ are the estimated unit and time weights.

Host-only subseries (countries that hosted but did not win) are excluded from the donor pool to avoid contamination from hosting effects. Standard errors are computed using block bootstrap with 1,000 replications, where blocks are defined at the unit level.

\subsection{Outcome Variables}

The analysis examines five main outcome variables, all measured as year-over-year growth rates in percentage points:
\begin{enumerate}
\item \textbf{GDP}: Gross domestic product (chain-linked volume, rebased, USD PPP)
\item \textbf{Consumption}: Final consumption expenditure (chain-linked volume, rebased, USD PPP)
\item \textbf{Investment}: Gross fixed capital formation (chain-linked volume, rebased, USD PPP)
\item \textbf{Exports}: Exports of goods and services (chain-linked volume, rebased, USD PPP)
\item \textbf{Imports}: Imports of goods and services (chain-linked volume, rebased, USD PPP)
\end{enumerate}

\subsection{Robustness Checks and Extensions}

To assess the robustness and generalizability of the findings, I conduct several additional analyses:

\begin{enumerate}
\item \textbf{The 2022 World Cup:} I examine Argentina's 2022 victory as an out-of-sample test, checking whether the patterns documented for earlier tournaments persist in the most recent edition.

\item \textbf{Finalist and semi-finalist effects:} I investigate whether countries reaching the final or semi-finals but not winning experience similar economic effects. This helps distinguish between the specific effect of winning versus the broader effect of strong tournament performance.

\item \textbf{Early elimination effects:} I examine whether there is a reverse effect for countries eliminated early in the tournament, particularly for nations with high football expectations where early exits might negatively impact national sentiment.

\item \textbf{Alternative specifications:} I vary the event window length and control variable specifications to assess sensitivity.
\end{enumerate}

\subsection{Implementation}

All analyses are conducted in R. Event study regressions are estimated using the \texttt{fixest} package, which efficiently handles high-dimensional fixed effects. SDID estimates are obtained using the \texttt{synthdid} package, which implements the methodology of \citet{Arkhangelsky2021}. Results are cross-validated using the \texttt{lfe} package as an alternative estimator. All code and data are provided in the electronic appendix for full reproducibility.
