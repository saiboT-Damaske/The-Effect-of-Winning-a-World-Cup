This section outlines the empirical strategy used to estimate the causal effect of winning the FIFA World Cup on economic outcomes. The analysis follows the approach of \citet{Mello2024}, implementing both event study and synthetic difference-in-differences methods.

\subsection{Identification Strategy}

The identification strategy exploits the quasi-random nature of World Cup final outcomes. While reaching the final is not random, the outcome of a single elimination match is influenced by factors---referee decisions, weather, injuries, and luck---that are plausibly exogenous to a country's economic fundamentals. This quasi-randomness, combined with fixed effects that absorb permanent country differences and common time shocks, allows us to interpret the estimated coefficients as causal effects.

\subsection{Event Study Specification}

The event study specification follows the theoretical framework from Section~\ref{background}, with controls added as in \citet{Mello2024}:

\begin{equation}
\Delta_4 \ln(\text{GDP}_{c,t}) = \alpha_c + \mu_t + \sum_{l=-16}^{-2} \beta_l \cdot \mathbf{1}[\text{WIN}^l_{c,t}] + \sum_{l=1}^{16} \beta_l \cdot \mathbf{1}[\text{WIN}^l_{c,t}] + \theta \cdot \text{HOST}_{c,t} + \zeta \cdot \ln(\text{GDP}_{c,t-4}) + \varepsilon_{c,t}
\label{eq:event_study_applied}
\end{equation}

The outcome $\Delta_4 \ln(\text{GDP}_{c,t})$ is year-over-year GDP growth for country $c$ in quarter $t$, computed as the log-difference between current and year-ago GDP. The indicators $\text{WIN}^l_{c,t}$ equal one when country $c$ is $l$ quarters from a World Cup victory. Country fixed effects $\alpha_c$ absorb time-invariant characteristics, while quarter-year fixed effects $\mu_t$ control for global economic conditions. The host indicator $\text{HOST}_{c,t}$ captures hosting effects, and the fourth lag of log GDP controls for convergence dynamics. The reference category is $l = 0$, the quarter when finals occur.

The sample covers 48 countries from 1961-Q1 to 2021-Q4, yielding 8,637 observations after dropping missing values. Six countries are winners with 10 observed victories: Germany (1974, 1990, 2014), Italy (1982, 2006), France (1998, 2018), Brazil (2002), Spain (2010), and England 1966 (mapped to UK aggregate). Standard errors are clustered at the country level. The model is estimated using the \texttt{fixest} package in R.

\subsection{Synthetic Difference-in-Differences Specification}

The SDID analysis focuses on the six World Cups from 1998 to 2018. Following \citet{Arkhangelsky2021}, I construct 10-quarter subseries around each tournament, spanning $q = -7$ to $q = +2$ relative to Q2 of the World Cup year. Each country contributes one subseries per tournament, creating a stacked panel.

The SDID estimator computes:

\begin{equation}
\hat{\tau}^{sdid} = \left( \bar{Y}_{tr,post} - \sum_{q=-7}^{0} \hat{\lambda}_q Y_{tr,q} \right) - \sum_{n \in \mathcal{C}} \hat{\omega}_n \left( \bar{Y}_{n,post} - \sum_{q=-7}^{0} \hat{\lambda}_q Y_{n,q} \right)
\label{eq:sdid_applied}
\end{equation}

where $\bar{Y}_{tr,post}$ is the post-treatment average for treated subseries (winners in quarters $q = 1, 2$), $\hat{\omega}_n$ are unit weights on control subseries, and $\hat{\lambda}_q$ are time weights on pre-treatment quarters.

The treated subseries are France 1998, Brazil 2002, Italy 2006, Spain 2010, Germany 2014, and France 2018. Host-country subseries (other than winners) are excluded from the donor pool. After requiring balanced panels, the estimation sample contains 274 subseries (6 treated, 268 control) with 2,740 observations. Standard errors are computed via 1,000 bootstrap replications using the \texttt{synthdid} package.

\subsection{Outcome Variables}

The analysis examines five outcome variables, all measured as year-over-year growth rates in percentage points: GDP, private consumption, government consumption, gross fixed capital formation (investment), and exports. All variables come from OECD Quarterly National Accounts, chain-linked volume in constant USD PPP. This decomposition follows Mello's mechanism analysis and helps identify whether any aggregate effect operates through demand channels or trade channels.

\subsection{Robustness Checks}

The robustness analysis includes alternative lag specifications for the event study (no lag, lags 3 through 6, and all lags simultaneously), matched control samples selected by GDP or population nearest neighbors, and comparison of winners to runners-up to distinguish winning effects from tournament visibility. I also implement the heterogeneity-robust estimators of \citet{Sun2021} and conduct placebo tests by reassigning treatment to control countries and shifting treatment timing to pre-event periods.

All analysis is conducted in R with data processing in Python. The complete pipeline is provided in the electronic appendix.
