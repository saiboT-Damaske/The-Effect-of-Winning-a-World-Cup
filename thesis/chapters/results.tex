% ====================================================================
\subsection{Replication of Mello (2024): Event Study}
\label{sec:es_gdp}
% ====================================================================

I estimate the event study regression for GDP specified in Equation~\eqref{eq:mello_event_study}.  The dependent variable is year-over-year log GDP growth $\Delta_4 \ln \text{GDP}_{c,t}$ in percentage points.  The 32 relative-time indicators $\text{WIN}^l_{c,t}$ for $l \in \{-16, \ldots, -1, +1, \ldots, +16\}$ capture the dynamic effect of winning the World Cup, with endpoints binned and the event quarter $l = 0$ omitted as reference.  A hosting dummy, the fourth lag of logged GDP as a convergence control, and country and quarter fixed effects complete the specification.  Standard errors are clustered at the country level.

As described in Section~\ref{sec:sample_construction}, the year-over-year growth rates are computed on the full 1960+ panel before trimming to the estimation window.  Additionally, the convergence control $\ln y_{c,t-4}$ is pre-computed from the untrimmed panel and joined to the estimation sample.  The final estimation sample contains 8,633 country-quarter observations, four fewer than Mello's 8,637.  The difference in observation counts is due to minor discrepancies in the underlying data extracts.

\subsubsection{GDP Results}
\label{sec:es_gdp_results}

Table~\ref{tab:event_study_comparison} presents selected coefficients from our replication alongside \citeauthor{Mello2024}'s Table~2 results.  The full set of 32 coefficients, including exact $p$-values and differences, is provided in the appendix (Table~\ref{tab:es_full_comparison}).

\begin{table}[H]
\centering
\caption{Event study for GDP: replication versus \citet{Mello2024}}
\label{tab:event_study_comparison}
\small
\begin{tabular}{l rr rr}
\toprule
 & \multicolumn{2}{c}{\textbf{Replication}} & \multicolumn{2}{c}{\textbf{Mello (2024)}} \\
\cmidrule(lr){2-3} \cmidrule(lr){4-5}
                     & Coeff. & (SE)    & Coeff. & (SE)    \\
\midrule
\multicolumn{5}{l}{\textit{Controls}} \\
$\ln \text{GDP}_{t-4}$ & $-1.364$** & (0.613) & $-1.368$** & (0.588) \\
Host                    & $-0.542$   & (0.484) & $-0.591$   & (0.545) \\[4pt]
\multicolumn{5}{l}{\textit{Pre-treatment}} \\
$l = -16$ (binned)      &  0.771 & (0.671) &  0.640 & (0.673) \\
$l = -8$                &  0.219 & (0.665) &  0.098 & (0.695) \\
$l = -4$                &  0.073 & (0.515) & $-0.107$ & (0.535) \\
$l = -1$                &  0.071 & (0.227) &  0.125 & (0.206) \\[4pt]
\multicolumn{5}{l}{\textit{Post-treatment}} \\
$l = +1$                &  0.325   & (0.269) &  0.454*  & (0.246) \\
$l = +2$                &  0.597   & (0.387) &  0.683*  & (0.370) \\
$l = +3$                &  0.305   & (0.319) &  0.233   & (0.335) \\
$l = +4$                &  0.287   & (0.319) &  0.140   & (0.317) \\
$l = +8$                &  0.066   & (0.775) & $-0.314$ & (0.985) \\
$l = +16$ (binned)      &  0.073   & (0.463) & $-0.109$ & (0.477) \\
\midrule
Observations            & \multicolumn{2}{c}{8,633} & \multicolumn{2}{c}{8,637} \\
Adj.\ $R^2$            & \multicolumn{2}{c}{0.444} & \multicolumn{2}{c}{---}   \\
Within $R^2$ (Mello)    & \multicolumn{2}{c}{---}   & \multicolumn{2}{c}{0.423} \\
\bottomrule
\multicolumn{5}{l}{\footnotesize *\,$p<0.10$,\; **\,$p<0.05$,\; ***\,$p<0.01$.\; Clustered SEs at the country level.}
\end{tabular}
\end{table}

The pre-treatment coefficients ($l < 0$) are uniformly small and statistically insignificant, consistent with the parallel trends assumption.  In the post-treatment period, GDP growth rises by 0.33 percentage points at $l = +1$ and 0.60\,pp at $l = +2$.  Both point estimates are positive but fall short of conventional significance thresholds ($p = 0.23$ and $p = 0.13$, respectively; see appendix Table~\ref{tab:es_full_comparison} for exact $p$-values).  The effect dissipates by the third quarter and subsequent coefficients fluctuate around zero.  The convergence control ($-1.364$, $p = 0.031$) is consistent with conditional $\beta$-convergence; the hosting dummy ($-0.542$, $p = 0.269$) is negative but insignificant.

\subsubsection{Comparison with Mello (2024)}

The replication tracks the original results closely.  The convergence control matches to three significant digits ($-1.364$ vs.\ $-1.368$), and all pre-treatment coefficients lie within one standard error.  The key post-treatment coefficients at $l = +1$ and $l = +2$ show the same positive pattern, though our estimates are slightly attenuated (0.33 and 0.60 vs.\ 0.45 and 0.68).  This attenuation---0.13\,pp and 0.09\,pp respectively---moves both coefficients below the 10\% significance threshold that \citet{Mello2024} reports as marginally significant.  The differences are modest and likely reflect minor discrepancies in the underlying data and estimation procedures.

Figure~\ref{fig:es_gdp_comparison} overlays the full coefficient paths.  The replication (blue) tracks the original (red) closely across all 32 relative-time periods, with the largest visible discrepancy occurring in the long-run post-treatment lags ($l \geq +5$) where both sets of estimates are statistically indistinguishable from zero.

\begin{figure}[H]
\centering
\includegraphics[width=0.95\textwidth]{event_study_gdp_comparison.png}
\caption{GDP event study: replication vs.\ \citet{Mello2024}}
\label{fig:es_gdp_comparison}
\begin{tablenotes}
\textit{Notes}: Solid lines show point estimates; shaded areas show 95\% confidence intervals based on country-clustered standard errors.  Blue = replication ($N = 8{,}633$), red = \citeauthor{Mello2024} ($N = 8{,}637$).
\end{tablenotes}
\end{figure}

The core qualitative finding is robust: winning the World Cup is associated with a transient rise in GDP growth of roughly 0.3--0.7 percentage points in the two quarters following the event, which dissipates by the third post-treatment quarter.


% ====================================================================
\subsection{GDP Component Event Studies}
\label{sec:es_components}
% ====================================================================

To examine the channels through which a World Cup victory might affect aggregate GDP, I estimate Equation~\eqref{eq:mello_event_study} separately for each of the five expenditure components: private consumption, government consumption, gross fixed capital formation, exports, and imports.  In each regression, both the dependent variable $\Delta_4 \ln y_{c,t}$ and the convergence control $\ln y_{c,t-4}$ use the respective component.  All other specification details remain identical to the GDP model.  The component regressions have 8,589 observations, 44 fewer than GDP.  The difference in observation counts is due to minor discrepancies in the underlying data extracts.

Figure~\ref{fig:es_all_comparison} compares the coefficient paths of all six features between our replication and \citeauthor{Mello2024}'s Table~2 (GDP) and Table~A3 (components).  A detailed comparison table with selected post-treatment coefficients for all six features is provided in Table~\ref{tab:es_all_features_comparison} in the appendix.

\begin{figure}[H]
\centering
\includegraphics[width=0.95\textwidth]{event_study_all_features_comparison.png}
\caption{Event study comparison across GDP components: replication vs.\ \citet{Mello2024}}
\label{fig:es_all_comparison}
\begin{tablenotes}
\textit{Notes}: Each panel shows the full coefficient path for one GDP component.  Blue = replication, red = \citeauthor{Mello2024}.  Shaded areas are 95\% confidence intervals.  Component regressions use 8,589 observations (replication) vs.\ 8,549 (\citeauthor{Mello2024}); GDP uses 8,633 vs.\ 8,637.
\end{tablenotes}
\end{figure}


Across all six components, the replication coefficients track \citeauthor{Mello2024}'s closely.  The convergence controls, which are the most precisely estimated parameters, match within a few hundredths for every component (e.g.\ capital formation: $-7.816$ vs.\ $-7.795$; imports: $-6.146$ vs.\ $-6.056$).  The observation count difference between our component regressions (8,589) and Mello's (8,549) reflects minor discrepancies in the underlying data extracts.

The component-level results identify \textbf{exports} as the primary channel: at $l = +2$, export growth increases by 5.81\,pp ($p = 0.032$) in our replication---the only statistically significant post-treatment coefficient at the 5\% level across all features.  \citet{Mello2024} reports a comparable 5.12\,pp for the same coefficient, significant at the 10\% level.  The $l = +3$ export coefficient is also marginally significant in our estimates (4.27\,pp, $p = 0.081$).  These magnitudes are roughly an order of magnitude larger than the aggregate GDP effect, consistent with the effect being concentrated in a single component.

The hosting dummy is positive and significant for exports in our replication ($1.89$, $p = 0.033$), indicating that hosting a World Cup independently boosts export growth, likely through tourism-related demand and heightened international visibility.  \citeauthor{Mello2024} reports a similar direction (1.25) but without significance.

\textbf{Private consumption} and \textbf{government consumption} show no meaningful response in either our replication or Mello's.  Both $l = +1$ and $l = +2$ coefficients are small and insignificant, ruling out a household-spending channel despite the documented spikes in consumer confidence following sporting victories.  \textbf{Capital formation} is imprecisely estimated due to high investment volatility, while \textbf{imports} show a suggestive delayed effect at $l = +4$ (4.18\,pp in our replication, $p = 0.057$; 4.33\,pp in Mello, $p < 0.05$), consistent with increased demand accompanying the export surge.

In summary, the component analysis confirms \citeauthor{Mello2024}'s finding that the World Cup GDP effect operates primarily through the trade channel.  Where the observation counts are closely matched, the coefficient differences between our replication and the original are small---typically well within one standard error---and both analyses point to the same qualitative conclusions.

% COMMENT: SDID results will be added once the SDID scripts are updated.
