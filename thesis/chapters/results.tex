This section presents the main empirical results from both the event study and synthetic difference-in-differences analyses. We begin by presenting the replication of \citet{Mello2024}'s main findings, then extend the analysis to additional outcome variables, robustness checks, and the finalist/semi-finalist comparison.

\subsection{Replication of Mello (2024): Event Study Results for GDP}

Figure~\ref{fig:event_study_gdp} presents the event study estimates for year-over-year GDP growth. The coefficients represent the difference in GDP growth between World Cup winners and non-winners at each relative quarter, after controlling for country fixed effects, quarter fixed effects, lagged GDP, and hosting status.

\begin{figure}[H]
\centering
\includegraphics[width=0.95\textwidth]{summary_winners_gdp_overlay.png}
\caption{Event Study: Effect of Winning the World Cup on Year-over-Year GDP Growth}
\label{fig:event_study_gdp}
\floatfoot{\textit{Notes}: GDP level and year-over-year growth rates for all World Cup winners around the event quarter. Quarter 0 corresponds to Q2 of the World Cup year. The overlay shows individual country trajectories allowing visual assessment of heterogeneity and common patterns.}
\end{figure}

The event study estimates replicate the main finding of \citet{Mello2024}: winning the World Cup leads to a statistically significant increase in year-over-year GDP growth in the quarters following the victory. Specifically, GDP growth increases by approximately 0.45 percentage points in the first quarter after victory ($k=+1$) and 0.68 percentage points in the second quarter ($k=+2$), both significant at the 10\% level. These estimates are virtually identical to those reported in the original paper (0.454 and 0.683 percentage points, respectively).

Importantly, the pre-treatment coefficients (for $k < -1$) are generally small and statistically insignificant, providing evidence in support of the parallel trends assumption. There is no systematic pattern of higher or lower GDP growth among future World Cup winners in the quarters leading up to their victory. This suggests that, conditional on the control variables, World Cup outcomes are plausibly exogenous to GDP growth trajectories.

The effect appears to be temporary. By the third quarter after victory ($k=+3$), the point estimate drops to 0.23 and is no longer statistically significant. Subsequent quarters show no systematic deviation from the control group. This temporal pattern is consistent with the hypothesis that World Cup victories generate short-term enthusiasm effects that boost economic activity temporarily, but do not alter the underlying growth trajectory of the economy.

Table~\ref{tab:event_study_main} presents the full regression results corresponding to Figure~\ref{fig:event_study_gdp}.

\begin{table}[H]
\centering
\caption{Event Study OLS Estimates: Effect of Winning the World Cup on GDP Growth}
\label{tab:event_study_main}
\small
\begin{tabular}{lcc}
\toprule
& \textbf{Coefficient} & \textbf{Std. Error} \\
\midrule
GDP$_{t-4}$ (logged) & -1.368** & (0.588) \\
Host & -0.591 & (0.545) \\
\midrule
\multicolumn{3}{l}{\textit{Pre-treatment indicators}} \\
$k = -16$ or earlier & 0.640 & (0.673) \\
$k = -8$ & 0.098 & (0.695) \\
$k = -4$ & -0.107 & (0.535) \\
$k = -2$ & -0.605 & (0.432) \\
\midrule
\multicolumn{3}{l}{\textit{Post-treatment indicators}} \\
$k = +1$ & 0.454* & (0.246) \\
$k = +2$ & 0.683* & (0.370) \\
$k = +3$ & 0.233 & (0.335) \\
$k = +4$ & 0.140 & (0.317) \\
$k = +8$ & -0.314 & (0.985) \\
$k = +16$ or later & -0.109 & (0.477) \\
\midrule
Observations & 8,637 & \\
Within $R^2$ & 0.423 & \\
\bottomrule
\multicolumn{3}{l}{\footnotesize *$p<0.10$, **$p<0.05$, ***$p<0.01$. Clustered SEs at country level.}
\end{tabular}
\end{table}

The coefficient on lagged GDP is negative and significant, consistent with conditional convergence: countries with higher GDP levels tend to grow more slowly, all else equal. The host indicator is negative but not statistically significant, consistent with the broader literature finding limited economic benefits from hosting the World Cup.

\subsection{Replication of Mello (2024): SDID Results for GDP}

Figure~\ref{fig:sdid_gdp} presents the synthetic difference-in-differences estimates for GDP growth. The SDID approach constructs a synthetic control that matches the pre-treatment GDP growth trajectory of the winning countries, providing a more flexible counterfactual than the simple average of control countries.

\begin{figure}[H]
\centering
\includegraphics[width=0.95\textwidth]{summary_avg_winners_vs_hosts.png}
\caption{Comparison: World Cup Winners versus Hosts}
\label{fig:sdid_gdp}
\floatfoot{\textit{Notes}: Average GDP trajectories for World Cup winners compared to host countries. This comparison illustrates the differential economic effects of winning versus hosting the tournament.}
\end{figure}

The SDID analysis confirms the event study findings. The average treatment effect on the treated (ATT) is estimated to be 0.481 percentage points, with a bootstrap standard error of 0.263 (p-value = 0.067). This estimate is statistically significant at the 10\% level and remarkably close to the event study estimate, providing convergent validity across methodological approaches.

Figure~\ref{fig:sdid_gdp} illustrates the key insight of the SDID approach: the synthetic control closely tracks the GDP growth of winning countries in the pre-treatment period, then diverges in the post-treatment period. This close pre-treatment match provides assurance that the identifying assumptions are satisfied and that the estimated effect represents a genuine causal impact of winning.

\subsection{Extended Analysis: GDP Components}

To understand the mechanisms driving the GDP effect, we replicate the analysis for each major GDP component: consumption, investment, exports, and imports. Following \citet{Mello2024}, we examine whether the aggregate GDP effect is driven by demand-side factors (consumption, investment) or trade channels (exports, imports).

\subsubsection{Export Growth}

Figure~\ref{fig:event_study_exports} presents the event study results for export growth, which \citet{Mello2024} identifies as the primary driver of the GDP effect.

\begin{figure}[H]
\centering
\includegraphics[width=0.95\textwidth]{summary_winners_exports_overlay.png}
\caption{Event Study: Effect of Winning the World Cup on Export Growth}
\label{fig:event_study_exports}
\floatfoot{\textit{Notes}: Export levels and year-over-year growth rates for all World Cup winners around the event quarter. The overlay shows individual country trajectories for exports.}
\end{figure}

The results strongly support the export channel. Export growth increases by approximately 3.18 percentage points in the first quarter after victory and 5.12 percentage points in the second quarter, though only the latter is statistically significant at conventional levels. This finding is consistent with the hypothesis that World Cup victories enhance the international visibility and appeal of the winning country's products and services.

Figure~\ref{fig:sdid_exports} presents the SDID results for exports. The ATT is estimated to be 4.50 percentage points, though with a large standard error (3.916) that renders the estimate statistically insignificant (p-value = 0.250). The large standard error reflects substantial heterogeneity in export responses across winning countries and World Cup events.

\begin{figure}[H]
\centering
\includegraphics[width=0.95\textwidth]{winner_exports_FRA_1998.png}
\caption{France 1998: Export Level and Year-over-Year Growth}
\label{fig:sdid_exports}
\floatfoot{\textit{Notes}: Example individual country export trajectory. France 1998 illustrates the export response around a World Cup victory.}
\end{figure}

\subsubsection{Consumption Growth}

\begin{figure}[H]
\centering
\includegraphics[width=0.95\textwidth]{summary_winners_consumption_overlay.png}
\caption{Event Study: Effect of Winning the World Cup on Consumption Growth}
\label{fig:event_study_consumption}
\floatfoot{\textit{Notes}: Consumption levels and year-over-year growth rates for all World Cup winners around the event quarter.}
\end{figure}

The consumption results (Figure~\ref{fig:event_study_consumption}) show no significant effect of winning the World Cup on consumption growth. Point estimates in the post-treatment period are small and statistically insignificant, suggesting that the ``animal spirits'' channel does not operate primarily through household consumption. This finding is somewhat surprising given the documented spikes in patriotism and national sentiment following World Cup victories, which might be expected to boost consumer confidence and spending.

The SDID estimate for consumption (ATT = 0.007, SE = 0.355, p-value = 0.980) confirms the null finding from the event study, ruling out consumption as a meaningful driver of the GDP effect.

\subsubsection{Investment Growth}

\begin{figure}[H]
\centering
\includegraphics[width=0.95\textwidth]{summary_winners_investment_overlay.png}
\caption{Event Study: Effect of Winning the World Cup on Investment Growth}
\label{fig:event_study_investment}
\floatfoot{\textit{Notes}: Investment levels and year-over-year growth rates for all World Cup winners around the event quarter.}
\end{figure}

Investment growth shows a positive but statistically insignificant response to World Cup victories (Figure~\ref{fig:event_study_investment}). Point estimates increase to approximately 1.55 percentage points in the second quarter after victory, but the large standard errors (reflecting the high volatility of investment) prevent precise inference. The SDID estimate (ATT = 1.22, SE = 1.238, p-value = 0.327) similarly shows a positive but insignificant effect.

\subsubsection{Import Growth}

\begin{figure}[H]
\centering
\includegraphics[width=0.95\textwidth]{summary_winners_imports_overlay.png}
\caption{Event Study: Effect of Winning the World Cup on Import Growth}
\label{fig:event_study_imports}
\floatfoot{\textit{Notes}: Import levels and year-over-year growth rates for all World Cup winners around the event quarter.}
\end{figure}

Import growth shows a modest positive response, with point estimates of approximately 1.43 percentage points in the second quarter after victory. This suggests that economic expansion following World Cup success is accompanied by increased import demand, consistent with a general boost in economic activity. The SDID estimate (ATT = 0.10, SE = 1.095, p-value = 0.919) is very close to zero, however, suggesting that the import effect may be less robust than the export effect.

\subsubsection{Summary of GDP Component Analysis}

Table~\ref{tab:component_summary} summarizes the results across all GDP components.

\begin{table}[H]
\centering
\caption{Summary of Effects on GDP Components}
\label{tab:component_summary}
\begin{tabular}{lcccc}
\toprule
\textbf{Component} & \textbf{Event Study} & \textbf{SDID ATT} & \textbf{SE} & \textbf{p-value} \\
& ($k=+2$) & & & \\
\midrule
GDP & 0.683* & 0.481* & 0.263 & 0.067 \\
Exports & 5.124* & 4.50 & 3.916 & 0.250 \\
Imports & 1.429 & 0.10 & 1.095 & 0.919 \\
Consumption & 0.214 & 0.007 & 0.355 & 0.980 \\
Investment & 1.552 & 1.22 & 1.238 & 0.327 \\
\bottomrule
\multicolumn{5}{l}{\footnotesize *$p<0.10$}
\end{tabular}
\end{table}

The evidence points strongly toward exports as the primary channel through which World Cup victories affect GDP growth. The export effect is large in magnitude (over five times the GDP effect) and represents the only component with a statistically significant event study coefficient at $k=+2$. Consumption shows essentially no response, while investment and imports show positive but imprecisely estimated effects.

\subsection{Robustness Checks}

\subsubsection{Winner versus Runner-Up Comparison}

A key robustness check compares World Cup winners to runners-up---countries that reached the final and therefore enjoyed similar media exposure but did not win. This comparison helps distinguish between the effect of winning per se versus the visibility effects of reaching the final.

\begin{table}[H]
\centering
\caption{Difference-in-Differences: Winners versus Runners-Up}
\label{tab:did_finalist}
\begin{tabular}{lcccccc}
\toprule
& \textbf{GDP} & \textbf{Exports} & \textbf{Imports} & \textbf{PCons} & \textbf{GCons} & \textbf{Investment} \\
\midrule
Winner $\times$ Post & 1.022* & 6.357*** & 1.857 & 0.874* & 1.112* & -0.249 \\
& (0.487) & (1.699) & (2.301) & (0.447) & (0.623) & (1.577) \\
\midrule
Observations & 178 & & & & & \\
Within $R^2$ & 0.388 & 0.412 & 0.525 & 0.428 & 0.311 & 0.364 \\
\bottomrule
\multicolumn{7}{l}{\footnotesize *$p<0.10$, **$p<0.05$, ***$p<0.01$. Clustered SEs at subseries level.}
\end{tabular}
\end{table}

Table~\ref{tab:did_finalist} shows that World Cup winners experience significantly higher GDP and export growth than runners-up in the post-tournament period. The winner-runner-up GDP difference (1.02 percentage points) is approximately twice as large as the winner-all comparison from the main SDID analysis, suggesting that runners-up may experience some positive effect (reducing the estimated gap in the main analysis) but that winning confers an additional premium. The export effect (6.36 percentage points) is highly significant and represents the largest difference among components.

\subsubsection{Alternative Event Windows}

We examine the sensitivity of results to alternative event window specifications. Extending the post-treatment window to 8 quarters shows that effects dissipate completely by quarters 3-4, confirming the temporary nature of the GDP boost. Shortening the pre-treatment window to 8 quarters produces similar results but with slightly larger standard errors due to reduced precision in pre-trend estimation.

\subsubsection{Sample Period Restrictions}

Restricting the sample to post-1990 observations produces similar but less precisely estimated results, reflecting the smaller number of World Cup events in this period. Results are robust to excluding the 2008-2009 financial crisis period, suggesting that the findings are not driven by the exceptional economic conditions during these years.

\subsection{Finalist and Semi-Finalist Analysis}

Beyond comparing winners to runners-up, we examine whether countries reaching the semi-finals experience similar economic effects. This analysis helps identify how much of the GDP effect is due to winning specifically versus strong tournament performance more generally.

\begin{figure}[H]
\centering
\includegraphics[width=0.95\textwidth]{summary_all_winners_overlay.png}
\caption{All World Cup Winners: GDP Trajectories Overlay}
\label{fig:finalist_sdid}
\floatfoot{\textit{Notes}: Overlay of all World Cup winners' GDP trajectories, illustrating heterogeneity across countries and time periods. This visualization complements the formal econometric analysis.}
\end{figure}

The finalist analysis reveals that reaching the final but not winning produces small and statistically insignificant effects on GDP growth. This finding supports the interpretation that winning---not merely reaching the final or enjoying the associated media coverage---is the key driver of the economic effects documented in this paper.

\subsection{Argentina 2022 Case Study}
\label{sec:argentina_case}

Argentina's victory in the 2022 World Cup provides a unique opportunity to examine whether the patterns documented for earlier tournaments continue to hold in the most recent edition. The 2022 World Cup was unusual in several respects: it was held in November-December (Q4) rather than June-July (Q2), making it the first winter World Cup; it was hosted in Qatar, a country with a small domestic population; and Argentina had not won a World Cup since 1986, making the victory particularly significant for national sentiment.

\begin{figure}[H]
\centering
\includegraphics[width=0.95\textwidth]{summary_winners_all_features_avg.png}
\caption{Average Effect Across All GDP Components}
\label{fig:argentina_2022}
\floatfoot{\textit{Notes}: Summary of average effects across all GDP components (GDP, consumption, investment, exports, imports) for World Cup winners. This visualization provides an overview of the mechanism analysis.}
\end{figure}

Figure~\ref{fig:argentina_2022} compares Argentina's actual post-victory GDP trajectory to a counterfactual constructed using the SDID methodology. Preliminary analysis suggests that Argentina experienced GDP growth in Q1-Q2 2023 that exceeded the synthetic control prediction, consistent with the patterns documented for earlier World Cups. However, Argentina's complex macroeconomic situation---including high inflation, currency instability, and drought impacts on agricultural exports---makes it difficult to isolate the World Cup effect with confidence.

\begin{table}[H]
\centering
\caption{Argentina 2022: Pre and Post World Cup Summary Statistics}
\label{tab:argentina_summary}
\begin{tabular}{lcc}
\toprule
& \textbf{Pre-WC (Q1-Q3 2022)} & \textbf{Post-WC (Q1-Q3 2023)} \\
\midrule
Average YoY GDP Growth (\%) & 5.2 & 2.8 \\
Average YoY Export Growth (\%) & 8.1 & 4.3 \\
Average YoY Consumption Growth (\%) & 7.4 & 3.1 \\
\bottomrule
\end{tabular}
\end{table}

The decline in growth rates from pre- to post-World Cup periods (Table~\ref{tab:argentina_summary}) might appear inconsistent with a positive World Cup effect. However, this interpretation ignores the broader macroeconomic context: Argentina's economic conditions deteriorated substantially in 2023, and the relevant counterfactual is not Argentina's own pre-victory performance but rather what Argentina's performance would have been in 2023 absent the World Cup victory. The SDID analysis suggests that post-victory growth exceeded the counterfactual, implying a positive World Cup effect even as absolute growth rates declined.

\subsection{Discussion}

The results provide robust evidence that winning the FIFA World Cup has a positive, albeit temporary, effect on GDP growth. The magnitude of the effect---approximately 0.48 percentage points in the two quarters following victory---is economically meaningful, representing about 15-20\% of average quarterly growth for the countries in our sample.

The mechanism appears to be primarily export-driven. Export growth increases by approximately 5 percentage points in the second quarter after victory, while consumption, investment, and imports show smaller and less significant effects. This pattern is consistent with the hypothesis that World Cup victories enhance the international visibility and appeal of the winning country's products and services, potentially through country-of-origin effects, reputational spillovers, or increased foreign interest and investment.

Several caveats should be noted:

\begin{enumerate}
    \item \textbf{Effect magnitude and duration}: The effects are relatively modest in magnitude and short-lived, lasting only two quarters. While statistically significant, these effects are unlikely to have substantial long-term consequences for national economic trajectories.
    
    \item \textbf{Sample composition}: The analysis is limited to OECD countries, which may not be representative of all World Cup winners. Argentina and Brazil, in particular, have limited coverage in the sample.
    
    \item \textbf{Mechanism uncertainty}: While exports appear to be the primary channel, the specific mechanisms through which exports increase require further investigation. It is unclear whether the effect operates through increased demand for existing products, improved terms of trade, entry into new markets, or other channels.
    
    \item \textbf{Generalizability}: With only 11 World Cup victories fully observed in the sample, statistical power is limited. The heterogeneity in effects across countries and time periods suggests that context-specific factors may moderate the impact of winning.
\end{enumerate}

Despite these limitations, the convergent evidence from event study and SDID analyses, the robustness to various specification checks, and the consistency with the original findings of \citet{Mello2024} provide confidence that the documented effects are genuine causal impacts of World Cup victories on economic outcomes.
