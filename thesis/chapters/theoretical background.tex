% Theoretical Background Section
% This section covers the theoretical foundations of event study methodology
% and synthetic difference-in-differences estimation

\subsection{Event Study Methodology}

Event study methodology is a statistical technique widely used in economics and finance to measure the causal effect of a specific event on an outcome variable of interest. The fundamental idea behind event studies is to compare the behavior of the outcome variable before and after the event occurs, while controlling for what would have happened in the absence of the event \citep{MacKinlay1997}.

In the context of panel data, an event study specification typically takes the form:

\begin{equation}
Y_{it} = \alpha_i + \lambda_t + \sum_{k=-K}^{-1} \beta_k \mathbf{1}[E_{it} = k] + \sum_{k=1}^{L} \beta_k \mathbf{1}[E_{it} = k] + \varepsilon_{it}
\end{equation}

where $Y_{it}$ is the outcome variable for unit $i$ at time $t$, $\alpha_i$ represents unit fixed effects, $\lambda_t$ represents time fixed effects, and $E_{it}$ is the relative time to the event (with $E_{it} = 0$ denoting the event period). The coefficients $\beta_k$ for $k < 0$ capture pre-trends, while $\beta_k$ for $k > 0$ capture post-event effects. The indicator function $\mathbf{1}[\cdot]$ equals one when the condition is met and zero otherwise.

The identification assumption underlying event studies is that, conditional on unit and time fixed effects, the timing of the event is exogenous to the outcome variable. This assumption allows us to interpret the estimated coefficients as causal effects \citep{Borusyak2021}.

\citet{Sun2021} provide a comprehensive framework for event study estimation, emphasizing the importance of proper specification of the relative time indicators and the choice of the omitted category. A common practice is to omit the period immediately before the event ($k = -1$) to serve as the reference category, allowing all other coefficients to be interpreted relative to this baseline period.

\subsection{Synthetic Difference-in-Differences}

Synthetic Difference-in-Differences (SDID) is a recently developed methodology that combines the strengths of synthetic control methods and difference-in-differences estimation \citep{Arkhangelsky2021}. SDID addresses several limitations of traditional difference-in-differences approaches, particularly when treatment effects are heterogeneous across units or when the parallel trends assumption is violated.

The SDID estimator constructs a synthetic control unit by finding weights that minimize the pre-treatment difference between the treated and control units. Formally, for a panel dataset with $N$ units observed over $T$ periods, where treatment occurs at time $T_0$, the SDID estimator solves:

\begin{equation}
\hat{\tau}^{SDID} = \arg\min_{\tau, \omega, \lambda} \sum_{i=1}^{N} \sum_{t=1}^{T_0-1} \left( Y_{it} - \omega_i - \lambda_t - \tau D_{it} \right)^2
\end{equation}

subject to constraints on the weights $\omega_i$ and time effects $\lambda_t$, where $D_{it}$ is a treatment indicator.

The key innovation of SDID is its two-way weighting scheme: it assigns weights to both control units (to construct a synthetic control) and to pre-treatment periods (to emphasize periods that are more predictive of post-treatment outcomes). This dual weighting mechanism helps ensure that the synthetic control closely matches the treated unit's pre-treatment trajectory, thereby strengthening the parallel trends assumption \citep{Arkhangelsky2021}.

\citet{Ben-Michael2021} extend the SDID framework to settings with staggered treatment adoption, providing a robust estimator that can handle multiple treatment groups and varying treatment timing. The SDID estimator has been shown to have favorable finite-sample properties and can provide more accurate treatment effect estimates than traditional difference-in-differences when the parallel trends assumption is questionable.

The asymptotic properties of SDID are established under regularity conditions on the data generating process. \citet{Arkhangelsky2021} show that under appropriate conditions, the SDID estimator is consistent and asymptotically normal, with standard errors that can be estimated using a block bootstrap procedure.

In the context of this thesis, SDID is particularly well-suited for analyzing the economic effects of World Cup victories because:
\begin{enumerate}
    \item The treatment (winning the World Cup) occurs at different times for different countries, creating a staggered treatment design.
    \item Pre-treatment trends may vary across countries, making the parallel trends assumption of standard difference-in-differences potentially problematic.
    \item The synthetic control component allows us to construct a counterfactual that closely matches each winner's pre-victory economic trajectory.
\end{enumerate}

\subsection{Comparison with Mello (2024)}

\citet{Mello2024} employs both event study and SDID methodologies to examine the effect of winning the FIFA World Cup on GDP growth. The paper finds that winning the World Cup increases year-over-year GDP growth by at least 0.48 percentage points in the two subsequent quarters, with the effect primarily driven by enhanced export growth.

This thesis aims to replicate and extend the analysis of \citet{Mello2024}, examining the robustness of these findings through:
\begin{itemize}
    \item Replication of the main event study and SDID specifications
    \item Extension to additional outcome variables (consumption, investment, imports)
    \item Robustness checks using alternative specifications and sample restrictions
    \item Analysis of finalist and semi-finalist countries to examine whether similar effects exist for teams that perform well but do not win
\end{itemize}