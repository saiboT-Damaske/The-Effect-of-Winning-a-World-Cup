\subsection{The Impact of Large Sport Events}

\subsubsection{Social and Psychological Effects}

There is a vast amount of research documenting the impact that large sporting events can have on individuals at a psychological level. When national teams compete on the world stage, citizens experience heightened patriotism, pride, and shared identity. These effects are particularly pronounced during mega-events like the FIFA World Cup, where simultaneous mass viewership---reaching approximately 5 billion cumulative viewers for the 2022 tournament \citep{FIFA2022}---amplifies the collective emotional experience across entire populations.

Research documents both positive and negative social effects of major sporting events. \citet{Mutz2013}, using panel survey data collected during the 2012 UEFA European Championship, shows temporary increases in patriotism following tournament performance that dissipate within weeks. \citet{Kersting2007}, comparing survey responses across South Africa and Germany, documents similar patterns for national pride. \citet{DepetrisChauvin2020} provide rigorous causal evidence using survey data linked to African Cup of Nations matches, finding that national team victories shift citizens' identification toward national identity and increase inter-ethnic trust. \citet{Billings2013}, analyzing longitudinal survey data collected during the 2012 Olympics, find associations between Olympic media consumption and nationalistic attitudes. On the negative side, \citet{RosenzweigZhou2021}, using a natural experiment around the 2014 World Cup, show that sports victories can increase negative attitudes toward outgroups, and \citet{Bertoli2017}, employing time-series analysis of interstate disputes, demonstrates spillovers into international conflicts.

Beyond attitudes, mega-events generate substantial intangible value for host populations. \citet{DolanKavetsos2019}, using life satisfaction data from large-scale surveys, find increased subjective well-being from hosting the Olympics. \citet{GibsonWalker2014} document ``psychic income'' among host populations through pre- and post-event surveys in South Africa. \citet{Zhou2009}, surveying Beijing residents, report widespread agreement that hosting major events increases local pride and community cohesion.

\subsubsection{Economic Behavior and Willingness to Pay}

The psychological significance of sporting success translates into measurable economic valuations. Studies employing contingent valuation methods consistently find that citizens are willing to pay substantial amounts for their national team's success. \citet{Wicker2012}, using survey experiments in Germany, estimate a willingness-to-pay of approximately €4.26 per capita for a gold medal at international football tournaments. \citet{Hallmann2013}, comparing contingent valuation responses for football versus Olympic success, document that willingness-to-pay for football achievements substantially exceeds that for Olympic medals among German respondents. \citet{Bakkenbuell2018}, surveying German citizens before the 2014 World Cup, estimate a collective willingness-to-pay of approximately €1.46 billion for Germany winning the tournament. These substantial valuations suggest that football success, particularly at the World Cup, may generate economic consequences through consumer sentiment and spending behavior.

\subsubsection{Economic Effects of Hosting}

A widely covered topic in media and academic discourse is the economic effect of hosting large sporting events like the FIFA World Cup or the Olympic Games. Governments and organizing committees routinely commission studies projecting billions in economic impact, yet it often remains unclear whether the substantial public expenditures on stadiums, infrastructure, and security actually yield a positive return for host countries.

The academic literature reaches predominantly skeptical conclusions. For the Olympic Games, \citet{BillingHolladay2012}, using a synthetic control approach, find limited long-term economic effects of hosting. \citet{LiBlakeThomas2013}, employing computable general equilibrium modeling for Beijing 2008, find positive but modest impacts. Interestingly, \citet{RoseSpiegel2011} show that bid submission alone---independent of actual hosting---generates trade increases, suggesting signaling effects that do not require the actual expenditure of hosting.

The FIFA World Cup literature reaches similar conclusions. \citet{BaadeMatheson2004}, analyzing employment and taxable sales in US host cities from the 1994 World Cup, estimate economic losses rather than gains. \citet{HagnMaennig2008}, using monthly employment data, find no measurable employment effects from Germany hosting the 1974 World Cup. \citet{Szymanski2010} reviews the broader literature and concludes that economic benefits from hosting are typically overstated in ex ante projections. \citet{LeeTaylor2005} and \citet{Peeters2014} examine tourism effects, finding that increased visitor spending rarely justifies infrastructure costs. Despite this evidence, ex ante projections from organizing bodies remain optimistic \citep{FIFA2024}.

Since the effects of hosting have been extensively researched and the consensus suggests limited economic benefits, this thesis focuses on a less-studied question: the economic effect of \textit{winning} major tournaments. For the theoretical assumption of winning to have a macroeconomic effect, the event must be sufficiently large to generate widespread attention and emotional engagement across the population. The event that can most likely generate such macroeconomic effects through emotional channels is the FIFA World Cup. As documented above, \citet{Wicker2012} and \citet{Hallmann2013} find that willingness-to-pay for football success substantially exceeds that for Olympic success, reflecting football's unique cultural significance, particularly in Europe and South America.

\subsubsection{The Economics of Winning}

Research specifically examining the economic effects of \textit{winning} the World Cup is remarkably scarce. Popular media has given attention to apparent patterns in post-victory economic performance. Forbes magazine described an apparent pattern of post-victory GDP contraction as the ``World Cup GDP Curse,'' observing that in six of the last seven tournaments, the winning country's economy contracted in the following year \citep{Forbes2014}. The article highlights cases like Spain 2010, where GDP declined after victory, and Germany 2014, where growth slowed following their triumph. Figure~\ref{fig:spain_2010} illustrates this pattern for Spain, showing GDP levels and year-over-year growth rates around the 2010 World Cup victory.

\begin{figure}[H]
\centering
\includegraphics[width=0.95\textwidth]{winner_ESP_2010.png}
\caption{GDP trajectory for Spain around the 2010 World Cup victory. The red dashed line indicates the World Cup quarter (Q2 2010). The blue line shows GDP levels, while the orange line displays year-over-year growth rates.}
\label{fig:spain_2010}
\end{figure}

However, such descriptive analyses lack appropriate counterfactual construction. Observing that Spain's GDP declined after 2010 does not establish that winning the World Cup \textit{caused} this decline---Spain was already in the midst of a severe economic crisis stemming from the housing market collapse and European debt crisis. The relevant question is what would have happened to Spain's economy \textit{if} it had not won, requiring proper econometric methods to construct counterfactual trajectories.

\citet{Fett2020}, in a working paper, examines the relationship between World Cup victories and GDP growth, finding positive effects in the post-1990 era. However, the study relies on simple before-after comparisons without constructing appropriate counterfactuals through synthetic control or matching methods.

Only one rigorous causal analysis exists. \citet{Mello2024}, published in the Oxford Bulletin of Economics and Statistics, provides a thorough econometric investigation using quarterly GDP data from OECD countries spanning 1961 to 2021. Employing both event-study and synthetic difference-in-differences methodologies, Mello finds that winning the World Cup increases year-over-year GDP growth by approximately 0.48 percentage points in the two subsequent quarters. The effect appears primarily driven by export growth rather than domestic consumption or investment, suggesting that World Cup victories enhance the international appeal and visibility of the winning country's products and services. The scarcity of rigorous research in this area, combined with the potential significance of Mello's findings, motivates the present thesis.
