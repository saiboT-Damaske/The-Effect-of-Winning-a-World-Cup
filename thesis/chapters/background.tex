\subsection{The Impact of Large Sport Events}

\subsubsection{Social and Psychological Effects}

An underlying theory assumption of this thesis is that broadcasting and the consumption of sports have an impact on people. There is a vast amount of research documenting this impact that large sporting events can have on individuals at a psychological level. When national teams compete on the world stage, citizens, who often take on the role as fans, experience heightened patriotism, pride, and shared identity. These effects are particularly pronounced during mega-events like the FIFA World Cup, where simultaneous mass viewership -- reaching approximately 5 billion cumulative viewers for the 2022 tournament \citep{FIFA2022} -- amplifies the collective emotional experience across entire populations.

The effect can be both positive and negative though. \citet{Mutz2013}, using a panel survey with fixed-effects regression during the 2012 European Championship, finds significant temporary increases in patriotism that fully dissipate within three weeks after elimination. \citet{Kersting2007}, comparing pre- and post-tournament surveys in South Africa and Germany around the 2006 World Cup, documents 15--20 percentage point increases in national pride. \citet{DepetrisChauvin2020}, employing difference-in-differences on Afrobarometer surveys matched to African Cup of Nations results, find that victories increase national identification by 4 percentage points and inter-ethnic trust by 3 percentage points. \citet{Billings2013}, using regression analysis on longitudinal surveys during the 2012 Olympics, find that each additional hour of viewing increases nationalism scores by 0.02 points. On the negative side, \citet{RosenzweigZhou2021}, exploiting random match outcomes as instrumental variables around the 2014 World Cup, show that victories increase anti-refugee sentiment by 8 percentage points, and \citet{Bertoli2017}, using negative binomial regression on dispute data from 1950--2007, finds that football victories increase the probability of initiating interstate disputes by 11\%.

Beyond attitudes, mega-events generate substantial intangible value for host populations. \citet{DolanKavetsos2019}, applying difference-in-differences to Eurobarometer life satisfaction data, find hosting the Olympics increases well-being by 0.06 points on a 4-point scale. \citet{GibsonWalker2014}, surveying South African residents before and after the 2010 World Cup, find that intangible benefits like pride and excitement -- termed ``psychic income'' -- were the strongest predictor of public support for hosting. \citet{Zhou2009}, analyzing survey responses from 600 Beijing residents, report that 89\% agreed hosting increased local pride.

\subsubsection{Economic Behavior and Willingness to Pay}

The psychological significance of sporting success translates into measurable economic valuations. Studies employing contingent valuation methods consistently find that citizens are willing to pay substantial amounts for their national team's success. \citet{Wicker2012}, using a double-bounded dichotomous choice survey experiment with 1,064 German respondents, estimate a willingness-to-pay of €4.26 per capita for a football gold medal. \citet{Hallmann2013}, comparing contingent valuation responses across 2,027 Germans, find WTP for the 2012 European Championship (€6.30) exceeds WTP for Olympic gold (€3.50). \citet{Bakkenbuell2018}, applying interval regression to surveys of 1,000 Germans before the 2014 World Cup, estimate a collective WTP of €1.46 billion for winning. These substantial valuations suggest that football success may generate economic consequences through consumer sentiment.

\subsubsection{Economic Effects of Hosting}

A widely covered topic in media and academic discourse is the economic effect of hosting large sporting events like the FIFA World Cup or the Olympic Games. Governments and organizing committees routinely commission studies projecting billions in economic impact, yet it often remains unclear whether the substantial public expenditures on stadiums, infrastructure, and security actually yield a positive return for host countries.

The academic literature reaches predominantly skeptical conclusions. For the Olympic Games, \citet{BillingHolladay2012}, using synthetic control on US metro employment data, find no significant long-term effects of hosting. \citet{LiBlakeThomas2013}, employing a computable general equilibrium model for Beijing 2008, estimate a modest 0.24\% GDP increase. Interestingly, \citet{RoseSpiegel2011}, using gravity model estimation on bilateral trade data from 196 countries, show that bid submission alone generates an 18\% trade increase -- suggesting signaling effects independent of actual hosting.

The FIFA World Cup literature reaches similar conclusions. \citet{BaadeMatheson2004}, analyzing taxable sales and employment in 9 US host cities using OLS regression, estimate net losses of \$5.5--\$9.3 billion rather than the projected \$4 billion gain. \citet{HagnMaennig2008}, applying difference-in-differences to monthly employment data across German regions, find no measurable employment effects from hosting the 1974 World Cup. \citet{Szymanski2010} reviews the broader literature and concludes that benefits are typically overstated. \citet{LeeTaylor2005}, using input-output analysis for the 2002 Korea/Japan World Cup, find tourism benefits of \$307 million -- below the \$2.5 billion infrastructure cost. \citet{Peeters2014}, employing synthetic control for South Africa 2010, find no significant tourism increase. One exception is \citet{Fett2020}, who using fixed-effects OLS on host country GDP data from 1962--2010 finds a structural break at 1990: negative effects of $-4.6\%$ annually pre-1990 but positive effects of $+1.1\%$ post-1990, attributed to increased commercialization. Despite this mixed academic evidence, FIFA's official impact assessment for the 2026 World Cup projects \$40.9 billion in global GDP gains and 185,000 jobs in the US alone \citep{FIFA2024}.

Since the effects of hosting have been extensively researched and the consensus suggests limited economic benefits, this thesis focuses on a less-studied question: the economic effect of \textit{winning} major tournaments. For the theoretical assumption of winning to have a macroeconomic effect, the event must be sufficiently large to generate widespread attention and emotional engagement across the population. The most-watched sporting events globally include the FIFA World Cup (5 billion cumulative viewers in 2022, with 1.5 billion watching the final), the Summer Olympics (5 billion reached in Paris 2024), the ICC Cricket World Cup (over 1 trillion viewing minutes in 2023, though concentrated in South Asia), and the Super Bowl (approximately 125 million viewers globally) \citep{FIFA2022, IOC2024, ICC2023}. Among these, the FIFA World Cup stands out for combining massive global reach with intense national identification -- unlike the Olympics where attention is fragmented across dozens of sports and medal events, the World Cup final is a single match determining one national champion. As documented above, \citet{Wicker2012} and \citet{Hallmann2013} find that willingness-to-pay for football success substantially exceeds that for Olympic success, reflecting football's unique cultural significance, particularly in Europe and South America.

\subsubsection{The Economics of Winning}

Research specifically examining the economic effects of \textit{winning} the World Cup is remarkably scarce. Some media has given attention to apparent patterns in post-victory economic performance. For example, Forbes magazine described an apparent pattern of post-victory GDP contraction as the ``World Cup GDP Curse,'' observing that in six of the last seven tournaments (1986--2010), the winning country's economy contracted in the following year \citep{Forbes2014}. Figure~\ref{fig:winners_overlay} displays GDP growth trajectories for five of these seven winners. Argentina (1986) and Brazil (1994) are excluded as they do not have OECD quarterly accounts for this period. The pattern is heterogeneous: Spain 2010 shows a clear decline in GDP growth following victory, while Germany 1990 experienced continued growth. 

\begin{figure}[H]
\centering
\includegraphics[width=0.95\textwidth]{last_7_winners_gdp_overlay.png}
\caption[YoY GDP growth trajectories for five of the last seven World Cup winners (1986--2010)]{Year-over-year GDP growth trajectories for five of the last seven World Cup winners (1986--2010). Argentina 1986 and Brazil 1994 are not included due to OECD data availability. The red dashed line indicates the World Cup quarter (Q2), and the shaded region marks the post-victory period.}
\label{fig:winners_overlay}
\end{figure}

However, such descriptive analyses lack appropriate counterfactual construction. Observing that Spain's GDP declined after 2010 does not establish that winning the World Cup \textit{caused} this decline. Spain was already in the midst of a severe economic crisis stemming from the housing market collapse and European debt crisis. Similarly, Germany's strong post-1990 growth reflected reunification dynamics rather than World Cup effects. The relevant question is what would have happened to each country's economy \textit{if} it had not won, requiring proper econometric methods to construct counterfactual trajectories.

Only one rigorous causal analysis of the effect of \textit{winning} exists. \citet{Mello2024}, published in the Oxford Bulletin of Economics and Statistics, provides a thorough econometric investigation using event-study and synthetic difference-in-differences on quarterly OECD data spanning 1961--2021. Mello finds that winning increases year-over-year GDP growth by at least 0.48 percentage points (SE = 0.26) in the two subsequent quarters. The effect appears driven by export growth (+4.5 pp in the synthetic DiD, though insignificant) rather than consumption or investment, suggesting victories enhance international visibility. The scarcity of research in this area, combined with the potential significance of Mello's findings, motivates the present thesis and I will proceed to replicate his analysis to see if the findings match.