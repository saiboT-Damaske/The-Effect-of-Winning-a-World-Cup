% ==============================================================================
% DATA CHAPTER
% ==============================================================================

\subsection{Data Sources}

The primary data source is the OECD Quarterly National Accounts (QNA) database, accessed via the \texttt{OECD} R package \citep{OECD_R_pkg}.\footnote{Source tables: B1\_GS1 (GDP), P31S14\_S15 (private consumption), P3S13 (government consumption), P51 (gross fixed capital formation), P6 (exports), P7 (imports). All series are chain-linked volumes, rebased, seasonally adjusted, and converted to US dollars using purchasing power parity (PPP) exchange rates. Population data are from OECD annual statistics, interpolated to quarterly frequency.}
This is the same data source used by \citet{Mello2024}, ensuring direct comparability.
The dataset covers 48 countries over the period 1960--2024, though the estimation sample is restricted to 1962--2021 after accounting for lag requirements and the sample window specified in the original paper.

The six macroeconomic variables are:
\begin{enumerate}[noitemsep]
    \item Gross Domestic Product (GDP)
    \item Private consumption expenditure
    \item Government consumption expenditure
    \item Gross fixed capital formation (investment)
    \item Exports of goods and services
    \item Imports of goods and services
\end{enumerate}
Compared to \citet{Mello2024}, who reports results for GDP, aggregate consumption, investment, exports, and imports, the present study disaggregates consumption into its private and government components.
Annual population figures supplement the dataset for descriptive purposes.

World Cup results---winners, runners-up, semi-finalists, and host countries---are compiled from official FIFA records for all editions from 1930 to 2022.\footnote{The full list of World Cup results is available in Table~\ref{tab:app_wc_results} in the Appendix.}


\subsection{Sample Construction}
\label{sec:sample_construction}

The estimation sample is constructed to closely replicate that of \citet{Mello2024}.
The dependent variable is the year-over-year log growth rate:
\begin{equation}
    Y_{it} = \ln X_{it} - \ln X_{i,t-4}
    \label{eq:yoy}
\end{equation}
where $X_{it}$ is the level of the economic variable for country $i$ in quarter $t$.
This transformation removes quarterly seasonality and approximates the year-over-year percentage growth rate.

Computing~\eqref{eq:yoy} requires four quarters of lagged data, which introduces a subtlety in sample construction.
For most countries, OECD series begin in 1960-Q1 and the paper's stated window starts in 1961-Q1, yet the first valid growth observation is 1962-Q1 (since it requires the lag at 1961-Q1, which is the first quarter in the sample window).
A small number of countries with later data availability---Brazil (1998-Q2), Argentina (1993-Q1), Russia (1995-Q1), Chile (1996-Q1)---use external OECD data before their stated start dates to compute the initial growth rates.
This asymmetric lag treatment is documented only implicitly in \citet{Mello2024}, and matching it exactly was essential for reproducing the published observation counts:

\begin{table}[htbp]
\centering
\caption{Observation counts: replication vs.\ \citet{Mello2024}}
\label{tab:obs_comparison}
\small
\begin{tabular}{lccc}
\toprule
 & \textbf{Winners} & \textbf{Non-winners} & \textbf{Total} \\
\midrule
\citet{Mello2024} & 1,295 & 7,342 & 8,637 \\
This study         & 1,295 & 7,338 & 8,633 \\
\bottomrule
\end{tabular}
\begin{tablenotes}
\small
\item \textit{Notes:} ``Winners'' denotes all observations from countries that won a World Cup during the sample period (BRA, DEU, ESP, FRA, GBR, ITA). The four missing non-winner observations are attributable to Russia's OECD series ending in 2020-Q4 in our data download rather than 2021-Q4.
\end{tablenotes}
\end{table}

Six countries contribute as World Cup winners: Brazil, Germany, Spain, France, the United Kingdom, and Italy, covering 10 distinct victory events between 1966 and 2018 (Table~\ref{tab:wc_events}).
Argentina won the World Cup in 1978, 1986, and 2022, but quarterly GDP data begin only in 1993-Q1; since the 2022 event falls outside the main estimation window, Argentina is classified as a control country here and examined separately as a case study.
Brazil's 1994 victory similarly cannot be included because it requires pre-event data availability four years prior to the event (1990-Q2), but Brazilian data begin only in 1998-Q2.

\begin{table}[htbp]
\centering
\caption{World Cup events in the estimation sample}
\label{tab:wc_events}
\small
\begin{tabular}{llllcc}
\toprule
\textbf{Year} & \textbf{Winner} & \textbf{Host} & \textbf{Event Q} & \textbf{Winner} & \textbf{Host} \\
 & & & & \textbf{in sample} & \textbf{in sample} \\
\midrule
1966 & England        & England        & Q2 & \checkmark & \checkmark \\
1970 & Brazil         & Mexico         & Q2 & --         & \checkmark \\
1974 & West Germany   & West Germany   & Q2 & \checkmark & \checkmark \\
1978 & Argentina      & Argentina      & Q2 & --         & --         \\
1982 & Italy          & Spain          & Q2 & \checkmark & \checkmark \\
1986 & Argentina      & Mexico         & Q2 & --         & \checkmark \\
1990 & West Germany   & Italy          & Q2 & \checkmark & \checkmark \\
1994 & Brazil         & USA            & Q2 & --         & \checkmark \\
1998 & France         & France         & Q2 & \checkmark & \checkmark \\
2002 & Brazil         & Japan / Korea  & Q2 & \checkmark & \checkmark \\
2006 & Italy          & Germany        & Q2 & \checkmark & \checkmark \\
2010 & Spain          & South Africa   & Q2 & \checkmark & \checkmark \\
2014 & Germany        & Brazil         & Q2 & \checkmark & \checkmark \\
2018 & France         & Russia         & Q2 & \checkmark & \checkmark \\
\midrule
2022 & Argentina      & Qatar          & Q4 & \multicolumn{2}{c}{\textit{Case study}} \\
\bottomrule
\end{tabular}
\begin{tablenotes}
\small
\item \textit{Notes:} Event quarter is Q2 for all editions except Qatar~2022 (Q4). ``--'' indicates insufficient OECD data to include the event. The sample contains 10 winner events and 14 host events. Japan and Korea co-hosted in 2002. Full results including runners-up and semi-finalists are in Table~\ref{tab:app_wc_results}.
\end{tablenotes}
\end{table}


\subsection{Descriptive Statistics}

Table~\ref{tab:summary_stats} reports summary statistics for all macroeconomic variables, comparing countries that won a World Cup during the sample period (``winners'') with those that did not (``non-winners'').
Winner countries are systematically larger: their average GDP is roughly 2.2 times that of non-winners, and their average population is about 1.4 times larger.
This reflects the fact that World Cup success tends to be concentrated among large, wealthy nations---only eight countries have ever won the men's FIFA World Cup.

\begin{table}[htbp]
\centering
\caption{Summary statistics for the event-study sample}
\label{tab:summary_stats}
\begin{tabular}{lcccc}
\toprule
 & \multicolumn{2}{c}{Winner} & \multicolumn{2}{c}{Non-winner} \\
\cmidrule(lr){2-3} \cmidrule(lr){4-5}
 & Mean (SD) & & Mean (SD) & t-test \\
\midrule
\textbf{1960�80} \\
\quad GDP (in thousands of 2015 US dollar millions) & 1292.76 (512.37) & & 536.91 (1236.34) & 11.99*** \\
\quad Population (in millions) & 54.24 (13.93) & & 26.03 (45.84) & 12.18*** \\
\quad GDP per capita & 23,187.93 (5,203.63) & & 22,219.72 (10,419.19) & 1.81* \\
\quad Year-on-Year GDP growth & 4.21 (3.00) & & 4.73 (3.56) & -2.69*** \\
\textbf{1980�2000} \\
\quad GDP (in thousands of 2015 US dollar millions) & 2197.33 (736.26) & & 949.47 (2130.99) & 11.74*** \\
\quad Population (in millions) & 61.15 (22.37) & & 41.27 (100.62) & 3.99*** \\
\quad GDP per capita & 36,315.83 (7,322.12) & & 31,503.05 (16,208.15) & 5.91*** \\
\quad Year-on-Year GDP growth & 2.42 (1.79) & & 3.27 (3.63) & -4.63*** \\
\textbf{2000�20} \\
\quad GDP (in thousands of 2015 US dollar millions) & 3109.93 (819.26) & & 1368.24 (3020.17) & 12.57*** \\
\quad Population (in millions) & 84.89 (50.69) & & 66.51 (197.17) & 2.03** \\
\quad GDP per capita & 43,245.80 (13,238.25) & & 39,187.65 (21,569.91) & 4.01*** \\
\quad Year-on-Year GDP growth & 0.98 (3.29) & & 2.58 (3.98) & -8.38*** \\
\textbf{Full sample} \\
\quad GDP (in thousands of 2015 US dollar millions) & 2277.67 (1040.89) & & 1057.08 (2493.32) & 17.48*** \\
\quad Population (in millions) & 68.23 (37.34) & & 49.65 (149.16) & 4.49*** \\
\quad GDP per capita & 35,020.24 (12,743.09) & & 33,077.68 (19,305.46) & 3.52*** \\
\quad Year-on-Year GDP growth & 2.52 (3.29) & & 3.39 (3.96) & -7.49*** \\
\quad Number of countries & 6 & & 42 &  \\
\quad Number of observations & 1315 & & 7422 &  \\
\bottomrule
\end{tabular}
\begin{tablenotes}
\small
\item Notes: Standard deviations in parentheses. *p<0.10, **p<0.05, ***p<0.01.
\end{tablenotes}
\end{table}

Notably, winner countries exhibit \textit{lower} average growth rates across all six variables.
Average $\Delta_4 \ln$ GDP growth is 2.38 for winners versus 3.24 for non-winners ($t = -8.61$).
The same pattern holds for consumption, investment, exports, and imports, consistent with the convergence phenomenon: larger, more developed economies tend to grow more slowly.
This underscores the importance of the causal identification strategy, since na\"ive comparisons would suggest a negative association between World Cup success and economic growth.

Figure~\ref{fig:avg_winners_hosts} plots the average $\Delta_4 \ln$ GDP growth trajectory around World Cup events, separately for winner and host countries.
While both groups show stable pre-event growth, the post-event paths diverge---winner countries display a mild uptick in the quarters immediately following the tournament, an observation that motivates the formal econometric analysis.

\begin{figure}[htbp]
\centering
\includegraphics[width=0.95\textwidth]{summary_avg_winners_vs_hosts.png}
\caption{Average $\Delta_4 \ln$ GDP growth around World Cup events: winners vs.\ hosts ($\pm$16~quarters). The dashed red line marks the event quarter (Q2 of the World Cup year).}
\label{fig:avg_winners_hosts}
\end{figure}

Figure~\ref{fig:all_features_avg} extends this comparison across all six macroeconomic variables for winner countries.
GDP, private consumption, and government consumption follow similar, relatively smooth trajectories.
Investment, exports, and imports are substantially more volatile, limiting the statistical power for detecting treatment effects in these series.

\begin{figure}[htbp]
\centering
\includegraphics[width=0.95\textwidth]{summary_winners_all_features_avg.png}
\caption{Average $\Delta_4 \ln$ growth by macroeconomic variable for World Cup winners ($\pm$16~quarters).}
\label{fig:all_features_avg}
\end{figure}


\subsection{Pre-Tournament Rankings}
\label{sec:rankings}

To contextualise the strength of World Cup participants and identify ``underperformance'' (a top-ranked team eliminated in the group stage), we compile pre-tournament rankings from two independent sources:

\begin{itemize}[noitemsep]
    \item \textbf{ELO ratings} from \texttt{eloratings.net} \citep{EloRatings}, available for all 16 World Cups from 1962 to 2022. ELO ratings are based on all historical match results and update continuously.
    \item \textbf{FIFA World Rankings} from the FIFA internal API,\footnote{Endpoint: \texttt{inside.fifa.com/api/ranking-overview}. The ranking series begins in December 1992, covering 8~World Cups (1994--2022).} which reflect a points-based system incorporating match results, opponent strength, and confederation weights.
\end{itemize}

The two ranking systems are strongly correlated among World Cup participants (Table~\ref{tab:rank_corr}).
Spearman rank correlations range from 0.63 (1998) to 0.90 (2018), with a clear upward trend indicating convergence over time.
When defining ``underperformance'' as a top-10 ranked team (among WC participants) being eliminated in the group stage, only 5 out of 128 team-tournament observations (1994--2022) received a different label depending on the ranking system used.
In all five cases, ELO ranked the team inside the top~10 while FIFA did not, reflecting ELO's greater weight on historical performance.\footnote{The five cases are: Colombia (1994), Russia (1994), Spain (1998), Croatia (2002), and Uruguay (2022). See \citet{EloRatings} for methodology.}

\begin{table}[htbp]
\centering
\caption{Rank correlation between ELO and FIFA rankings among World Cup participants}
\label{tab:rank_corr}
\small
\begin{tabular}{lccc}
\toprule
\textbf{WC Year} & \textbf{$N$ teams} & \textbf{Spearman $\rho$} & \textbf{Mean $|$rank diff$|$} \\
\midrule
1994 & 24 & 0.70 & 4.1 \\
1998 & 32 & 0.63 & 6.4 \\
2002 & 32 & 0.84 & 4.2 \\
2006 & 32 & 0.85 & 4.1 \\
2010 & 32 & 0.86 & 3.6 \\
2014 & 32 & 0.89 & 3.3 \\
2018 & 31 & 0.90 & 3.2 \\
2022 & 31 & 0.89 & 3.0 \\
\bottomrule
\end{tabular}
\begin{tablenotes}
\small
\item \textit{Notes:} Rankings are among World Cup participants only. Spearman $\rho$ is the rank correlation between ELO and FIFA pre-tournament rankings. FIFA rankings begin in 1992, so no comparison is available before 1994.
\end{tablenotes}
\end{table}


\subsection{Data Quality and Limitations}

Several limitations should be acknowledged.
First, the OECD database predominantly covers developed economies, meaning that some World Cup winners (Brazil before 2002, Argentina throughout) have limited or no coverage during their victory periods.
Second, time-series length varies across countries, with several Eastern European and emerging economies entering the sample only in the mid-1990s (see Table~\ref{tab:app_coverage}).
Third, economic data are subject to revision; this analysis uses the most recent vintage accessed in 2024.
Finally, using national-level aggregates may mask heterogeneous effects across regions or sectors.
Despite these limitations, the OECD data represent the most comprehensive source of standardised quarterly national accounts and ensure consistency with \citet{Mello2024}.
