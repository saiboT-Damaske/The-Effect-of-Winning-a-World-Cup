Everyone knows them. The images of the celebrations of fans in a country that has just won the most watched sports event in the world: the FIFA World Cup. Only taking place every four years, the top 32 football nations\footnote{Initially 13 in 1930, then 16 from 1934--1978, 24 from 1982--1994, then 32 up to 2022.} compete for the official title of best football country in the world. The FIFA World Cup as an event has drawn major attention throughout the later 20th century and with each tournament broadcast viewership records are broken, with the latest tournament having around 5 billion viewers \citep{FIFA2022}. In the victor-country, this event is always followed by nation-wide celebrations and victory tours of the players. Millions of citizens gather at major landmarks, bars are filled, and everyday life is positively shocked by euphoria. And when the music dies out and the streets have been cleaned up, what remains? The observation of these events sparks the research question of this master thesis: does winning the FIFA World Cup have a positive impact on a country's economy?

This question sits at the intersection of sports economics, behavioral economics, and macroeconomics. On the one hand, the intense emotional responses triggered by World Cup victories; the spikes in patriotism, the collective euphoria, and the surge of national pride suggest channels through which economic activity might be affected. Consumer confidence may rise, spending may increase, and the winning nation's international visibility may boost demand for its products abroad or attract touristic visitors. But does the fundamental productive capacity of an economy change because its football team lifted a trophy? One would mainly assume that any economic effect, if it exists, should be expected to be temporary and modest relative to the underlying growth trajectory.

While sports economics literature has produced extensive research on the effects of hosting mega-sporting events, the question of whether winning such events matters economically has received far less attention. 

Some literature even suggests that the effect could be negative. Forbes magazine famously described an apparent pattern of post-victory GDP contraction as the ``World Cup GDP Curse,'' claiming that in six of the last seven tournaments the winning country's economy contracted the following year \citep{Forbes2014}. However, such claims rely on simple comparisons of GDP time series without proper counterfactual analysis, i.e.\ they do not construct the relevant counterfactual of what GDP growth would have been absent the victory.

Only one study I could come across has rigorously investigated this question. \citet{Mello2024}, published in the Oxford Bulletin of Economics and Statistics, provides a thorough causal analysis of the economic effects of winning the World Cup using quarterly GDP data from OECD countries spanning 1961 to 2021 and employing both event-study and synthetic difference-in-differences methodologies. Mello finds that winning the World Cup increases year-over-year GDP growth by approximately 0.48 percentage points in the two subsequent quarters. He adds that this effect appears to be driven primarily by export growth rather than consumption or investment which one would intuitively expect and suggests that a World Cup victory enhances the international appeal of the winning country's products and services.

The existence of only one rigorous study on this topic sparked my interest in expanding research on this topic. On the one hand, I want to validate the results through an independent replication. On the other hand, the scarcity of research means there is substantial room to extend and deepen the understanding of these effects or to demonstrate that they are not robust and that this effect does not require further investigation. 

The primary objective of this thesis is therefore twofold. First, I independently replicate Mello's event study and synthetic difference-in-differences (SDID) specifications using freshly constructed OECD data, verifying that the main results are reproducible.  Given the close alignment found between my replication and the original estimates, further robustness exercises (alternative lag structures, matched samples) were not pursued; the near-identical results from independent data and code already provide strong evidence for the validity of his robustness checks.

Second, I extend the analysis along two dimensions.  On the positive side, I broaden the treatment group to finalists (runner-up countries) and semi-finalists (top-four finishers) to test whether the economic premium extends beyond winning.  On the negative side, I examine \emph{underperformers} which I define as top-10 ELO-rated countries eliminated in the group stage and investigate whether the mechanisms operate in reverse.  Both extensions are estimated with the same event study and SDID methodology applied to all six GDP components (gdp, private consumption, government consumption, investment, exports, imports).

The remainder of this thesis is organized as follows. Section~\ref{background} reviews the literature on the economic effects of sporting events. Section~\ref{theory} presents the theoretical foundations of event study and synthetic difference-in-differences methodology. Section~\ref{methodology} describes the empirical implementation, including the stacked SDID design and bootstrap inference. Section~\ref{data} details the data sources and sample construction.  Section~\ref{results} reports the replication results compared to \citet{Mello2024}, and Section~\ref{extensions} presents the finalist, semi-finalist, and underperformer extensions.  Section~\ref{conclusion} concludes the the thesis' findings and my reflections.
