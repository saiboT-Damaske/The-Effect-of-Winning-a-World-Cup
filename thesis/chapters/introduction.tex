<remove this, once I gave the instruction to update the introduction chapter>
Everyone knows them. The images of the celebrations of fans in a country that has just won the most watched sports event in the world: the FIFA World Cup. Only taking place every four years, the top 32 football nations\footnote{Initially 13 in 1930, then 16 from 1934--1978, 24 from 1982--1994, then 32 up to 2022.} compete for the official title of best football country in the world. The FIFA World Cup as an event has drawn major attention throughout the later 20th century and with each tournament broadcast viewership records are broken, with the latest tournament having around 5 billion viewers \citep{FIFA2022}. In the victor-country, this event is always followed by nation-wide celebrations and victory tours of the players. Millions of citizens gather at major landmarks, bars are filled, and everyday life is positively shocked by euphoria. And when the music dies out and the streets have been cleaned up, what remains? The observation of these events sparks the research question of this master thesis: does winning the FIFA World Cup have a positive impact on a country's economy?

This question sits at the intersection of sports economics, behavioral economics, and macroeconomics. On one hand, the intense emotional responses triggered by World Cup victories---the spikes in patriotism, the collective euphoria, the surge of national pride---suggest channels through which economic activity might be affected. Consumer confidence may rise, spending may increase, and the winning nation's international visibility may boost demand for its products abroad or attract touristic visitors. But does the fundamental productive capacity of an economy change because its football team lifted a trophy? One would mainly assume that any economic effect, if it exists, should be expected to be temporary and modest relative to the underlying growth trajectory.

While sports economics literature has produced extensive research on the effects of hosting mega-sporting events, the question of whether winning such events matters economically has received far less attention. 

Some literature even suggests that the effect could be negative. Forbes magazine famously described an apparent pattern of post-victory GDP contraction as the ``World Cup GDP Curse,'' claiming that in six of the last seven tournaments the winning country's economy contracted the following year \citep{Forbes2014}. However, such claims rely on simple comparisons of GDP time series without proper counterfactual analysis and they do not cover the investigation of counterfactuals, namely what would have happened if the country did not win <add econometric technical term for this>

Only one study I could come across has rigorously investigated this question. \citet{Mello2024}, published in the Oxford Bulletin of Economics and Statistics, provides a thorough causal analysis of the economic effects of winning the World Cup using quarterly GDP data from OECD countries spanning 1961 to 2021 and employing both event-study and synthetic difference-in-differences methodologies. Mello finds that winning the World Cup increases year-over-year GDP growth by approximately 0.48 percentage points in the two subsequent quarters. He adds that this effect appears to be driven primarily by export growth rather than consumption or investment which one would intuitively expect and suggests that a World Cup victory enhances the international appeal of the winning country's products and services.

The existence of only one rigorous study on this topic <sparked my interest in expanding research on this topic>. On the one hand, <I want to validate the results> through independent replication before it can be considered robust. On the other hand, the scarcity of research means there is substantial room to extend and deepen our understanding of these effects or to demonstrate that they are not robust and that this effect does not require further investigation. 

<state what else we will do, once we know that (finalist analysis, reversed effect of loosing, other features (big maybe)) update this pargraph and the next one accordingly>The primary objective of this thesis is therefore twofold. First, I verify the robustness of Mello's findings through careful replication of the main empirical specifications by independently implementing the event-study and SDID methods and comparing results.

Second, I extend the analysis in several important directions. I examine the effects on all major GDP components---consumption, investment, exports, and imports---to provide a comprehensive picture of how World Cup victories affect different aspects of economic activity<this Mello already did too>. I conduct additional robustness checks including alternative event window specifications and sample restrictions. I investigate whether countries reaching the final or semi-finals experience similar effects, which helps distinguish between the effect of winning versus strong tournament performance. Finally, I provide a case study of Argentina's 2022 World Cup victory, examining whether the patterns documented for earlier tournaments continue to hold for the most recent edition.

The remainder of this thesis is organized as follows. Section~\ref{background} provides the background literature on the effects of sporting events, covering research on emotional and social impacts, the economics of hosting versus winning, and the specific context of World Cup economics that fuelled the idea of this thesis. Section~\ref{theory}<this will be renamed to methodology for which I will give instructions on how to rewrite> presents the theoretical framework for causal inference, including event study methodology and synthetic difference-in-differences estimation. Section~\ref{methodology} describes the empirical methodology and estimation procedures <I replicated form Mello and will apply to investigate other effects too (like finalist and semi finalist and maybe bad performance>. Section~\ref{data} presents the data sources and sample construction <used in both analysis>. Section~\ref{results} reports the main results<is extensive becasue it shows my results while also comparing them to the ones from Mello>, and Section~\ref{conclusion} concludes with a discussion of findings and implications.
