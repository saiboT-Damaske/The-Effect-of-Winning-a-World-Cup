This thesis has replicated and extended the analysis of \citet{Mello2024}, examining the economic effects of winning the FIFA World Cup. The main findings can be summarized as follows:

\subsection{Summary of Findings}

The replication of Mello's main results confirms that winning the World Cup leads to a statistically significant increase in GDP growth in the quarters following victory. The effect is estimated to be approximately 0.48 percentage points, consistent with the original paper's findings.

The extended analysis reveals that this effect is primarily driven by increased export growth, supporting Mello's hypothesis that World Cup victories enhance a country's international visibility and the appeal of its products and services. However, the effects on other GDP components (consumption, investment, imports) are [describe findings].

The analysis of finalist and semi-finalist countries provides important insights: [describe findings and implications].

\subsection{Contributions}

This thesis makes several contributions to the literature:
\begin{enumerate}
    \item It provides an independent replication of Mello's findings, confirming the robustness of the main results.
    \item It extends the analysis to additional outcome variables, providing a more comprehensive picture of how World Cup victories affect different aspects of economic activity.
    \item It examines the effects for finalist and semi-finalist countries, helping to distinguish between the effect of winning versus strong performance.
    \item It provides a detailed case study of Argentina's 2022 victory, illustrating the methodology with recent data.
\end{enumerate}

\subsection{Limitations and Future Research}

Several limitations should be acknowledged:
\begin{itemize}
    \item The analysis is limited to OECD countries, which may not be representative of all World Cup winners.
    \item The mechanisms through which World Cup victories affect economic outcomes (particularly exports) require further investigation.
    \item The effects are relatively modest and short-lived, suggesting that World Cup victories are unlikely to have substantial long-term economic consequences.
\end{itemize}

Future research could:
\begin{itemize}
    \item Examine the mechanisms through which exports increase (e.g., through increased demand, improved terms of trade, or changes in product mix).
    \item Investigate heterogeneous effects across countries with different economic structures or levels of development.
    \item Extend the analysis to other major sporting events to assess whether similar effects exist.
    \item Use more granular data (e.g., sector-level or firm-level) to better understand the channels through which World Cup victories affect economic outcomes.
\end{itemize}

\subsection{Policy Implications}

While the economic effects of winning the World Cup are statistically significant, they are relatively modest in magnitude. This suggests that policymakers should not expect World Cup victories to substantially boost long-term economic growth. However, the finding that exports appear to benefit from increased international visibility may have implications for export promotion strategies more broadly.

The temporary nature of the effects also suggests that any economic benefits from World Cup victories are likely to be short-lived, reinforcing the importance of sound long-term economic policies rather than relying on sporting success to drive economic performance.