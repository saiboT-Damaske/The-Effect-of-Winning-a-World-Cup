This thesis replicates and extends the analysis of \citet{Mello2024}, who examines whether winning the men's FIFA World Cup boosts GDP growth. Using OECD quarterly data and implementing both event study and synthetic difference-in-differences (SDID) methodologies, Mello finds that winning the World Cup increases year-over-year GDP growth by at least 0.48 percentage points in the two subsequent quarters, with the effect primarily driven by enhanced export growth.

The primary objective of this thesis is to verify the robustness of Mello's findings through careful replication of the main empirical specifications. In addition, I extend the analysis in several directions:

\begin{enumerate}
    \item \textbf{Extended outcome variables}: While Mello focuses primarily on GDP and exports, I examine the effects on additional GDP components, including consumption, investment, and imports, to provide a more comprehensive picture of how World Cup victories affect different aspects of economic activity.
    
    \item \textbf{Robustness checks}: I conduct a series of robustness checks, including alternative event window specifications, sample restrictions, and placebo tests, to assess the sensitivity of the results to methodological choices.
    
    \item \textbf{Finalist and semi-finalist analysis}: I examine whether countries that reach the final or semi-finals (but do not win) experience similar economic effects. This analysis helps distinguish between the effect of winning versus the effect of strong performance in the tournament.
    
    \item \textbf{Argentina 2022 extension}: I provide a detailed case study of Argentina's 2022 World Cup victory, comparing actual economic outcomes with counterfactual trajectories estimated using the SDID methodology.
\end{enumerate}

The thesis is structured as follows: Section 2 provides the theoretical background on event study methodology and synthetic difference-in-differences estimation. Section 3 describes the empirical methodology and estimation procedures. Section 4 presents the data sources and sample construction. Section 5 reports the main results, including replications of Mello's findings and the extended analyses. Section 6 concludes with a discussion of the findings and their implications. 